\subsection{Windows Issues}
\label{sec:binauth-windows}

We showed previously the complexities and challenges in Section~\ref{sec:win-issue}.
We now discuss below some special problems of
Windows which make it more difficult to implement binary authentication
than in other operating systems such as Unix.
Windows NT (Server 2000, XP, Server 2003, Vista) is a microkernel-like
operating system. 
Programs are usually written for the Win32 API but these are
decomposed into microkernel operations. 
However, Windows is closed source --- only the Win32 API is documented
and not the microkernel API.
Our prototype makes use of both the documented and undocumented
kernel infrastructure. However, it is not possible to make any guarantees
on the completeness of the security mechanisms
(which would also be a challenge even if Windows was open source).
Some of the specific issues in Windows which we deal with are:
\begin{itemize}
\item {\bf Proliferation of Binary Types:}
It is not sufficient to ensure the integrity of {\tt EXE} files.
In Windows, binaries can have any file name extension, or even no extension.
Some of the most common extensions include {\tt EXE} (regular
executables), {\tt DLL} (dynamic linked libraries), {\tt OCX} 
(ActiveX controls), {\tt SYS} (drivers), {\tt DRV} (drivers)
and {\tt CPL} (control panel applets).
Unlike Unix, binaries cannot be distinguised by an execution flag.
Thus, without reading its contents,
it is not possible to distinguish a binary from any other file.

\item {\bf Complex Process Execution:}
A process is created using \code{CreateProcess()} which is a Win32 library
function. However, this is not a system call since Windows is a microkernel,
and in reality this is broken up at the native API into:
\code{NTCreateFile()}, \code{NTCreateSection()}, \code{NTMapViewOfSection()},
\code{NTCreateProcess()}, \code{NTCreateThread()}.
Notice that \code{NTCreateProcess()} at the microkernel level
performs only a small part of what is needed to run a process.
Due to this, it is more complex to incorporate mandatory authentication in Windows.

\item {\bf DLL loading:}
To load a DLL, a process usually uses the Win32 API, {\tt LoadLibrary()}.
However, this is broken up in a similar way to process execution above.
% In order to load a particular DLL, a process first calls
% {\tt NtCreateFile()} \footnote{A process usually
% just calls {\tt LoadLibrary()} Win32 API, but we refer to the Native APIs here.}
% to open that DLL file.
% It subsequently calls {\tt NtCreateSection()} to create a section object,
% and then pass the section object to {\tt NtMapViewOfSection()} to map the file
% into its address space.

%However, many system DLLs known as ``KnownDLLs" \cite{knowndlls},
%on which most executables usually depend on, are loaded differently.
%A registry key defines this set of DLLs.
%They are loaded during system startup by {\tt smss.exe}, and the corresponding section 
%objects are registered in the system so that they can be opened later by 
%calling {\tt NtOpenSection()}.
%This mechanism, although useful for performance reasons, 
%is known to have introduced a security problem in the past \cite{knowndlls-vul}.
% describes a vulnerability in which a user can inject
%a malicious DLL by polluting KnownDlls namespace without modifying any system DLLs.

\item {\bf Execute Permissions:}
Many code signing systems, particularly those on Linux \cite{apvrille2004digsig,doorn01signedexecutables},
implement binary loading by examining the execute permission bit 
in the access mode of file open system-call.
The same mechanism, however, does not work in Windows.
Windows programs often set their file modes in a more permissive manner.
Simply denying a file opening with execute mode set 
when its authentication fails, will cause 
many programs to fail which are otherwise correct on Windows.
Instead, we need to properly intercept the right API(s) with
correctly intended operation semantics to respect Windows behavior.
%We have observed that there are files opened with execute
%permission by Windows Explorer and many other programs but the files are never
%executed.
%In other words, these files are opened with unnecessary execute permissions.
%If these file-open actions are denied, then these programs will not function correctly.

%\item {\bf TOCTTOU:}
%An access control system may suffer from Time-of-Check-to-Time-of-Use
%(TOCTTOU) race condition bug problem \cite{tocttou}.
%In windows, a binary is write-locked during its execution.
%That means, in order to ensure executing the correct file,
%the file only needs to be authenticated at the beginning
%of the execution.
%When the file is from a network file server, the lock is also implemented in the
%file server.
%Thus it has the same semantics as executing a local file.

\end{itemize}

%\TODO{fix version stuff, add brief discussion that windows patches do
%not discuss what files and versions are changed etc}

%Compared to other open platforms, Windows potentially also makes 
%the issue of locating vulnerable software components more complicated. 
%A great deal of binaries created by Microsoft contain an internal file version,
%which is stored as the file's meta-data. After Windows Update operations,
%the name of the updated file may however remain unchanged.
%Thus, it is difficult to keep track of files changes with such embedded version information.
%More precisely, one cannot ensure whether a version of a program $P_i$ 
%remain vulnerable to an attack $A$.
%Without an explicit mechanism like our software naming,
%manual inspection of files therefore is a difficult and daunting task.
%The picture is even made more complicated given that much of 
%Windows infrastructure remains undocumented. 
%As such, it is extremely difficult for a typical administrator
%who examines vulnerability information from public advisories
%to trace through the system and pinpoint the exact affected components.

Compared to other open platforms, Windows potentially also makes 
the issue of locating vulnerable software components more complicated. 
A great deal of binaries created by Microsoft contain an internal file version,
which is stored as the file's meta-data. The Windows update process does not
indicate to the user which files are modified. 
Moreover, meta data of the modified file might still be kept the same.
Thus, it is difficult to keep track of files changes in Windows.
More precisely, one cannot ensure whether a version of a program $P_i$ 
remain vulnerable to an attack $A$.
It is rather difficult for a typical administrator
who examines vulnerability information from public advisories
to trace through the system and pinpoint the exact affected components.
Our software naming scheme, associates binaries with
their version and simplifies software vulnerability management.
