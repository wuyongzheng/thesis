\section{Related Work}
\label{sect:related}

Tripwire \cite{KS93} is one of the first to do file integrity protection
but is limited as it in user-mode program and 
checks file integrity off-line.
It does not provide any mandatory form of integrity checking and
there are many known attacks such as:
file modification in between authentication times,
and attacks on system daemons (e.g. cron and sendmail)
and system files that it depends on \cite{Arnold,SKO05}.

There a number of kernel level binary level authentication implementations.
These are mainly for Unix
such as DigSig \cite{digsig}, Trojanproof \cite{williams}
and SignedExec \cite{signedexec}, which modify the Unix kernel
to verify the executable's digital signature before program execution.
DigSig and SignedExec embed signatures within the the elf binaries.
For efficiency, DigSig employs a caching mechanism to avoid checking
binaries which have been verified already. The mechanism is similar 
to ours here but we need to handle the problems of Windows.
It appears that DigSig provides binary integrity authentication
but not binary location authentication.\footnote{Mechanisms based solely on
signatures embedded in the binaries do not have sufficient information
for binary location authentication.}
In this paper, we examine the implementation issues and tradeoffs
for Windows which is more complex and difficult than in Unix.

Authenticode \cite{authenticode} is Microsoft infrastructure
for digitally signing binaries.
% It was publicly announced in 1996 as part of Microsoft's IE 3.0
% and ActiveX technologies, but it ships as a standard part of Windows.
% It tells where binaries come from and checks the integrity of binaries.
In Windows versions prior to Vista, such XP with SP2,
it is used as follows:
\begin{enumerate}
\item During ActiveX installation:
Internet Explorer uses Authenticode to examine the ActiveX plugin 
and shows a prompt which contains the publisher's information
including the result of the signature check.
\item A user downloads a file using Internet Explorer:
If this file is executed using the Windows Explorer shell,
a prompt is displayed giving the signed the publisher's information.
Internet Explorer uses an NTFS feature called
Alternate Data Streams to embed the Internet zone information --in this case,
the Internet-- into the file.
The Windows Explorer shell detects the zone information and displays the prompt.
This mechanism is not mandatory and relies on the use of zone-aware programs,
the browser and GUI shell cooperating with each other.
Thus, it can be bypassed.
\end{enumerate}

Since Authenticode runs in user space,
it can be bypassed in a number of ways, e.g. from the command shell.
It is also limited to files downloaded using Internet Explorer.
Only the EXE binary is examined by Authenticode, but DLLs are ignored.
One possible attack is then to put malware into a DLL and then
execute it, e.g. with {\tt rundll32.exe}.
Furthermore, Authenticode relies heavily on digital certificates.
Checking Certificate Revocation Lists (CRL) may add extra delay 
including timeouts due to the need to contact CA.
In some cases, this causes significant slowdown.

The latest Windows Vista improves on signed checking
because User Account Control (UAC) can be configured for
mandatory checking of signed executables.
However, this is quite limited since the UAC mechanism only
kicks in when a process requests privileged elevation,
and for certain operations on protected resources.
UAC is not user friendly since there is a need for constant
interactive user approval.
% of elevation requests and operations.
% maybe no space to have \cite{vista-complaints}.
Vista does not seem to prevent the loading of unsigned DLLs and 
other non EXE binaries. 
The 32-bit versions of Windows (including Vista) do not checked whether
drivers are signed.
However, the 64 bit versions (XP, Server 2003 and Vista) require
all drivers to be signed (this may be too strict and restrict hardware choices).

The closest work on binary authentication in Windows is
the Emu system in by Schmid et al. \cite{EMU}.
They intercept process creation by intercepting the NtCreateProcess system
call. It is unclear whether they are able authenticate all binary code
since trapping at NtCreateProcess is not sufficient to deal with DLLs.
No performance benchmarks are given, so it is unclear how 
if their system is efficient.

% Since it is important to find out whether the performance penalty due to the 
% authentication is acceptable, this question thus remains unanswered there.
% In addition, our paper here addresses a much wider scope of 
% secure program distribution and execution, taking into account various involved 
% parties (OS, software developers, advisory centers).
%The proposed scheme calculates MD5 hash values of the binaries,
%which are used together with the binary name to uniquely identify an application.
%Each user on the system has an {\em execution control list} (ECL)
%that specifies which applications the user is allowed to execute.
%Although we share the same basic mechanism of in-kernel authentication checking with \cite{EMU},
%our paper here addresses a much wider scope of secure program distribution and execution.
%Firstly, we first ensure that the binary must come with a valid digital signature first
%before generating the MAC for faster integrity checking.
%Secondly, we aim to also find out how low the performance penalty due to the integrity
%checking in Windows environments can be, and whether such penalty is acceptable.
%As no benchmark results shown in \cite{EMU}, the question thus remains to be answered.
%Thirdly, due to our bigger scope, our scheme deals with other new techniques
%such as software naming for vulnerability checking,
%and addresses how various parties (OS, software developers, advisory centers)
%can take advantage of the benefits brought.

% Another approach is to check simply at the filesystem layer such as
% \cite{i3fs}.
% Files are digitally signed,
% and the signature is checked every time the file is accessed.
% This has the drawback that the overhead of authentication is applied
% to all files accessed. For example, in Windows checking for file open with
% execute permissions will lead to many unnecessary checks.
% \TODO{check if the reference summary is correct for i3fs and FS approach}

% On discovering any failure in integrity check,
% the file-system layer will immediately block access to the files and notify the administrator.
% Although such approach can be useful, we view that it is not really suitable
% for our setting in ensuring a wider scope of secure software distribution and execution,
% in which many external parties are involved together.
% 
% [plus the work from LNCS]
