\subsection{Conclusion}
\label{sect:conclusion}

We have shown a comprehensive system which authenticates both content and pathname
for Windows to ensure that only trusted binaries are executed. 
Unlike other operating systems, Windows poses significant challenges.
We show that it is possible to ensure that only trusted binaries can be loaded
from files for execution. 
This can also be combined with a simple software ID
scheme which simplifies binary version management, and dealing
with vulnerability alerts and patches.
Our system is lightweight and integrates well with PKI and 
trust mechanisms without
having to rely on them.
The overheads of our prototype are quite low when caching is is used.
In the case of workloads with heavy file modifications, an uncached strategy might be preferable.
The overheads are still low in this case, since the system overhead
will be dominated by I/O rather than binary authentication, so the overall binary authentication
would still be low as a percentage of overall system overhead. 
In summary, although this is a prototype, it significantly adds
to the security of any Windows system but at the same time is sufficiently
flexible so that it can be tailored for different usage scenarios.

% In this paper we have performed a study on the feasibility of mandatory
% software component authentication. Our approach uses a lightweight
% authentication technique using MACs. According to our benchmarks, we
% found that using a cache reduces the overhead as it reduces
% the number of repeated authentication requests. 
% Based on these benchmarks,
% we can conclude that our mandatory authentication can be added to Windows
% without significant overhead.
% We have also introduced the concept of ``software naming''. The
% idea behind this is to uniquely identify a file. We have mentioned a few
% applications for this idea.
% Given the security benefits of binary authentication and software naming,
% and only small overhead incurred even on complex OS like Windows,
% we envisage that the presented results can better convince 
% the OS and user community alike to start deploying them more universally 
% in realizing secure software distribution and execution.
