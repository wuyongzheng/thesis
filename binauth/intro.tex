% \subsection{Introduction}
% \label{sec:binauth-intro}

Malware such as viruses, trojan horses, worms, remote attacks, 
are a critical security threat today.
A successful malware attack usually also modifies
the environment (e.g. file system) of the compromised host.
Many of the system security problems such as malware stem from the fact that untrusted code is executed
on the system.
We can mitigate many of these problems by ensuring that code which is executed only comes
from trusted software providers/vendors and the code is executed in the correct context.
In this section, we show that this can be efficiently achieved even on complex
operating systems such
as Windows. Our system provides two guarantees:
(i) we only allow the execution of binaries (in the rest of the section, we refer
to any executable code stored in the file system as a {\em binary})
whose contents are already known and trusted --- we call this {\em authenticating binary integrity};
and (ii) as binaries are kept in files, the pathname of the file must 
match its content --- we call this {\em authenticating binary location}. 
Binary integrity authentication ensures that the binary has not been
tampered with, e.g. {\tt cmd.exe} is not a trojan.
Binary location authentication ensures that we are executing
the correct executable content. 
A following extreme example illustrates location authentication. 
Suppose the binary integrity of the shell and the file system format executables 
are verified. If an attacker swaps their pathnames,
then running a shell would cause the file system to be formatted.
In this section, we refer to {\em binary authentication} to mean when 
both the binary's data integrity and location are verified.

Most operating systems can prevent execution of code
on the stack due to buffer overflow, e.g. NX protection.
Combining stack protection with binary authentication
makes the remaining avenues for attack 
smaller and more difficult.
%i.e. return-to-libc attacks and function pointer overwriting
%are still possible.
Binary authentication is also beneficial because it is even more important
for the operating system to be protected against malicious drivers
and the loading of malware into the kernel.

Most work on binary integrity authentication is on Unix/Linux
\cite{apvrille2004digsig,williams2002anti,doorn01signedexecutables}.
However, the problem of malware is more acute in Windows.
There are also many types of executable code,
e.g. executables (.exe), dynamic linked libraries (.dll), 
ActiveX controls, control panel applets, and drivers. 
In this section, we focus on mandatory binary authentication of all forms
of executables in Windows. 
Binaries which fail authentication cannot be loaded, thus,
cannot be executed. We argue that binary authentication together with 
execute protection of memory regions (e.g. Windows data execution prevention) 
provides protection against most of the malware on Windows.

A binary authentication needs to be flexible to operate under different scenarios.
Our prototype signs binaries using a HMAC \cite{krawczyk1997rfc2104} which is more lightweight
than having to rely on PKI infrastructure, although it can also make use of it.
The authentication scheme additionally allows for other security benefits.
Not only is it important to authenticate software on a system but one also needs
to deal with the maintenance of the software over time.
Nowadays, the number of discovered vulnerabilities grows rapidly 
\cite{CERT-vul}.
This means that binaries on a system (even if they are authenticated) may
be vulnerable. This leads to a vulnerability management and patching problems.
We propose a simple software ID system leveraging on the binary
authentication infrastructure and existing infrastructures such as DNS and certificate authorities
to handle this problem.

Windows has the Authenticode mechanism \cite{authenticode}.
In Windows XP version and earlier, 
it alerts the users of the results of signature verification under
a few situations. However, it is not mandatory, and can be easily bypassed.
The Windows Vista UAC mechanism makes use of signed binaries but it only
deals with EXE binaries. It is also limited to privilege escalation situations.
One common drawback of existing Windows mechanisms is that they do not 
authenticate the binary location.
Moreover, requiring PKI infrastructure and certificates, 
we believe, is too heavy for a general purpose mechanism.
% The latest Windows Vista, through its User Account Control (UAC), 
% makes some improvement by always alerting the users of signature verification status,
% particularly if the user runs with Administrator privilege.
% Yet, many problems remain, such as concern over certificate processing and revocation overhead, 
% and constant interactive user approval dialog which may irritate or annoy the users.

The main contribution of this work is that we believe that it provides the first
comprehensive infrastructure for trusted binaries for Windows.
This is significant given that much of the problems of security on Windows stems from
inability to distinguish between trusted and untrusted software.
It provides mandatory authentication for the full range of binaries under Windows, and goes beyond
authenticated code in XP and Vista.
We also protect driver loading which gives increased kernel protection.
Our scheme provides mandatory driver authentication which 32-bit Windows does not,
and can be integrated with more flexible policies which 64-bit Windows does not support.
We also analyze the security of our system. 
Our benchmarking shows that the overhead of comprehensive binary
authentication can be quite low, around 2\%,
with a caching strategy.
% Caching however is not always better.
% Under intensive file modification usage scenarios, 
% an uncached strategy can be preferable.

% The section is organized as follows.
% Section~\ref{sec:binauth-windows} examines special issues in Windows.
% Section~\ref{sec:binauth-related} surveys related work.
% We describe our scheme and implementation in Section~\ref{sec:binauth-scheme}.
% Section~\ref{sec:binauth-cost} presents the benchmarks, and
% Section~\ref{sec:binauth-conclusion} concludes this section.
