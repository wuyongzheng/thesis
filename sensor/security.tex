% include some discussions on knowledgeable malware with the ability to 
% detect the presence of the user. This system is still useful on
% those situations (e.g. it imposes an upper bound to the malware).
%
% provide some discussions on the nature of the side
% channel that connects all devices with the monitor.

\subsection{Security Discussion}

We further discuss two security considerations.

\paragraph{Communication Channel}

In view that malware may reside in the hosts,
it is important that external sensors are able to communicate with
the monitor securely so that the hosts are unable to compromise
the integrity, authenticity and confidentiality of the communication.
One solution is to have a separate private network for the sensors,
e.g. a wireless sensor network with the external sensors as the nodes.
A cheaper implementation could tunnel encrypted communication
through the host but that would be susceptible to DOS attacks if
the host is compromised.
The privacy requirements and the need for a separate private
network fit well with wireless sensor networks equipped with
multi-modality sensors.
%There are many commercial wireless multi-sensor
%boards which fit our purposes, e.g.
%SBT80 from the EasySen
%\cite{EasySen} contains a number of sensors including infra-red,
%temperature and acoustic.

\paragraph{Knowledgeable Malware}

Let us consider the scenarios where
(i) the hosts are compromised by malware,
(ii) the malware knows the system architecture and the detection algorithms,
including the threshold values, and
(iii) the external sensors, routers and the monitor are not compromised.
Since the malware resides in the host, it may able
to accurately derive user presence using information from, for example,
the keyboard and mouse.
Based on the derived information, the malware can attempt
to carry out malicious activities while mimicking normal activities, in order
to evade detection.
However, note that to evade detection, the malware is
constrained by the detection threshold and can only utilize resources at a low
rate when the user is not present.
When the user is present, although the threshold is
higher, the malware risks of being detected by the user.
For example, the
user may notice the interactive response is slow on the machine.
Furthermore, user presence detection based on information accessible 
by the host is arguably less accurate than through specifically 
designed external environment
sensors which add more constraints on the malicious activities.
