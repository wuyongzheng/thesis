\section{Conclusion}
\label{sec:conclusion}

In this section, we proposed a framework that incorporates environment
information in securing host computers in a few ways: by using the
environment information as an additional source of information for
malware detection, or by integrating the environment information
with existing conditions in rate-control mechanisms and access
control policies.   We argued that, since the sensors are
``external'' with respect to the host, they are difficult to be
accessed and tampered by a compromised host. Furthermore, 
we present three prototype intrusion detection applications which show that
coarse information on user activities and resource usages is
sufficient to provide intrusion detection even with a compromised host.
It is also useful in expressing certain rate-control and access control
policies. We have also identified a few important requirements of
the sensors, in particular, the concerns of user-privacy and the
need of a secure channel. Thus, we have proposed a simple and
effective framework for security enhancement which is arguably safe
against compromise by attackers.

The framework also takes advantage of the growing popularity of
pervasive computing and sensor networks.  As the trend in cost of wireless
multi-modality sensors is decreasing, applications of our framework are
feasible for cost-effective deployment in the near future.
