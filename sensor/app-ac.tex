\section{Application to Access Control and Rate Control}
\label{sec:app-ac-rate}

Here we investigate how environment information can be useful in
controlling and allocating resources.  A security policy may require
administrators to be physically inside the machine room to access
server consoles and administrative tools. In order to mitigate
malware from sending spam email, we could implement rate control in
the router or the mail server, and the rate limit is based on user
presence. In both cases, environment information is used as a
condition to access certain resources. Note that the first case is
on access control, while the second is on rate control.

\subsection{Access Control}
\label{sec:app-ac}

Our framework can be used to implement location-based access
control. One location-based access control scheme was proposed by Ardagna et
al. \cite{ardagna2006supporting} to restrict access to certain resources based on
physical location of the user. In their work, the physical location
is obtained from mobile devices, such as mobile phones carried by
users. Our framework also implements location-based access
control, but in a different way. Their work adopts a user-centric
approach where the device is attached to the user and
user's location is measured in order to figure out
which resources can be accessed.
We adopt a resource-centric approach where the device is attached to
the resource and user presence is observed near the resource in order
to decide whether to grant the access.

Both approaches have advantages and disadvantages.
When there are more resources than users, the resource-centric approach
incurs more cost for the sensors.
However, in the scenarios we discuss, there are likely fewer resources
than users or it is closer to a 1:1 ratio.
A user-centric approach may not be able to determine location
reliably, e.g. GPS would not work indoors,
% \TODO{tampering is not an issue - commented out the old text}
% The user-centric approach not only determines user location,
% it can also determine, for example, user's velocity which
% can be useful in some applications.
% On the other hand, with a resource-centric approach,
% the sensors device is hard to be tampered by the user,
% because it is not possessed by the user.
% For example, with user-centric approach, a malicious user can hack
% the device to produce false location information;
% whereas with resource-centric approach, the sensoring device for
% a server is located in the server room and the malicious has to
% physically get access to the server room first.
% With user-centric approach, the sensoring device can be separated from
% the user an produce false location information,
% and it is hard to determine user location
% in some environment; 
whereas a resource-centric approach does not have to determine location.

Our access control policy not only incorporates user presence information
but also user activity information.
For example, the user activity information can be 
``the user has typed some keys''.
Enforcement is implemented both on the host and
router depending on the type of the resource. Enforcement implemented on
the host implicitly assumes that malware is not in control of the
host and the host decides the access based on environmental
information gathered by the monitor, but the malware cannot affect
access control policies at the environment level.
Thus, the monitor and host cooperate, which is different from the 
earlier IDS applications where the monitor operates independently from the host.
We give two access control policies to illustrate environment aware
access control:

\begin{enumerate}
\item {\em The user can execute the {\tt /usr/bin/sudo} program only if he
is sitting at the host.}
This policy is used to mitigate the problem of remote attacks - it requires
that there is a human present before the sudo operation is allowed.
This policy has to be enforced on the host.
A remote attacker who does not yet have control of the host would
be prevented from performing actions which could be used to infect the
operating system.

\item {\em The user can send email only if he is sitting at the computer
and has mouse or keyboard activity.}
This policy is used to prevent malware from sending email while the user
is away.
Since user activity is enforced in the policy, it also prevents sending
email while the user is idle, e.g. watching a movie.
The intuition is that the user must have performed some typing or
mouse action in order to send an email.
Unlike user presence, which can be thought of as being
a continuous signal, mouse and keyboard input are discrete events which may
happen at time points which do not overlap with the email sending time interval.
Thus, the precise meaning of the policy is
``An email can be sent at time $t$ if user presence is observed at that $t$
and there is mouse or keyboard input between $[t-\Delta, t+\Delta]$''.
This policy is enforced in the router or the mail server instead of
the host and thus provides enforcement even when the host is compromised.
\end{enumerate}

\subsection{Rate Control}
\label{sec:app-rate}

Previously we showed that abnormal resource usage when the user
is absent can be easily detected under several scenarios.
This can be easily extended to controlling the resource usage
rate in the case of activities involving external resources.
Two natural scenarios of external rate control are:
\begin{itemize}
\item {\bf Shaping Network Traffic} \\
To mitigate computers being used as bots to perform DDoS attacks,
the router can shape network traffic based on user presence information.
By limiting the traffic on flows, this becomes a form
of inverse quality of service, providing reduced quality when the user
is not present. If P2P programs are being used, this would
save some network bandwidth, e.g. if the host uses Skype and becomes
a Skype supernode, it could lose its supernode status under
the reduced network flow.


\item {\bf Sending Email} \\
Similarly, the emails could be simply denied when the user is absent.
If the threshold is above zero, then
for both webmail and SMTP, the router can simply
rate limit the protocol. Alternatively for an SMTP server, it could
quarantine email beyond the threshold. In both cases, outgoing spam emails
are rate limited possibly to zero.

The email rate can also be rate limited when the user is present.
The idea is that email is based on typing and/or mouse clicks.
This data can also be recorded using environmental sensors.
The email rate can then be based on a function of the detected
keystrokes and/or mouse clicks.
\end{itemize}

The framework can make resource rate limiting policies more flexible and usable.
The idea is to have feedback if a usage is too high for a resource.
External sensors can then be used as a secure channel to request a higher
resource rate.
For example, a monitor on the host can display the resource usage and
warn the user if he is reaching the limit.
The user can have something as simple as a button which is pressed
to function as the secure request
channel to obtain more network bandwidth or more emails.
A low resolution camera could also be employed as an out-of-band channel
to the monitor where the low resolution may mitigate privacy concerns.
For example, it would be easy for the user to give hand signals to
communicate to the monitor.
On the other hand, implementing
such a policy securely is difficult 
without the help of external environment information.

In the access and rate control application discussed above,
to control the resource utilization,
essentially we need to verify whether an entity is ``human''.
Instead of using external environment to infer user presence,
alternatively, a CAPTCHA could be used.
Using the environment information
has the advantage that the users are not interrupted by the
challenges issued by the graphical Turing test.

