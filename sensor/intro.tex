The previous chapter shows software monitors that execute in the {\em host},
which is shared with the monitored software.
We have showed their robustness under the assumption that the kernel is
authentic.
However, when the kernel is compromised, the {\em host-based} monitoring
cannot be trusted.
We propose {\em external monitoring}, where information is collected from
monitors external to the host, so that the information can be trusted
even in a compromized host.
We have shown\footnote{
The work is cooperatively done by three students,
Liming Lu, Jie Yu and me, with two supervisors Ee-Chien Chang and Roland Hock Chuan Yap.
My part of work is on the external monitoring framework, which
will be shown in this chapter.
} \cite{chang2010enhancing} that this can be used
to detect malware in a host if we correlate user interaction with host behaviour,
both of which are observable by our external monitor.
In addition, the information of user interaction can be used to guide
resource allocation, thus limiting the resource utilization of malware.

\section{Introduction}

Securing computers against malware is increasingly difficult today. 
Anecdotes abound that the survival time
of an unpatched PC running Windows XP connected to the Internet 
is in the order of minutes~\cite{Survival1,Survival2}.
Worms can be quite effective in infecting systems, e.g.
the Conficker worm~\cite{conficker6}
is estimated to have infected 6\% of computers on the Internet.

Often the goal of the attackers is to infect a host to make it part
of a botnet.
The malware may be mostly dormant until it is activated,
as such, it can be difficult to detect that the host is infected.
The detection problem is made worse since malware can exploit rootkit
techniques to hide its presence and also any activity.
For example, the Mebroot~\cite{Mebroot} rootkit infects the master boot record of the
hard disk allowing it to infect the Windows kernel during boot.
After that it hides the changes to the master boot record to make it
difficult for antivirus software to detect its presence.

Many security mechanisms are host-based -- either
part of the operating system or interact with it.
A primary exception is network-based security mechanisms which
analyze network data and traffic.
While there are some successes in detecting the presence or activity of
malware by network-based security mechanisms, there are many other
sources of information outside the host that are also useful for
making detection and security mechanisms more robust.

We propose a framework which
fuses external (with respect to the host machine)
environment information collected by external sensors
in decision making to have more robust and enhanced
security mechanisms.
We employ the framework to:
(i) enhance intrusion detection;
(ii) monitor and control resource usage;
and (iii) enhance access and rate control policies with
user presence and out-of-band feedback.
The advantage of our framework is that by using the external sensors,
we are able to give reliable security guarantees even when the host is
compromised, i.e. by a rootkit.
Our focus will be on protecting stationary hosts (e.g.
workstation) which users work on rather than servers controlled remotely
or mobile laptops.\footnote{
We remark that it is feasible to deal with mobile devices but the framework
will need to be more complex and will need to incorporate authentication
features, thus, for simplicity and clarity, we focus on non-mobile hosts.
}

We take advantage of the growth in pervasive computing and sensor technology
which provide (relatively) cheap sensors which can take a variety of physical
measurements.
These sensors are used in a variety of ways.
Firstly, the sensor can be used to measure resource usage on the host itself,
e.g. correlating CPU usage (the resource)
with CPU temperature (the sensor measurement).
Secondly, the sensors can be used to measure the interaction of the host
with the environment, e.g. network usage which also includes sending emails.
Thirdly, the sensors can be used to measure environmental information
which is orthogonal to the host, the most important being
using sensors to determine the presence or absence of
the user on the host as well as physical user activity such as keyboard usage.
By ``user'', we mean the human using the host and although
there may be more than one user, we simply say user.

We are also concerned with maintaining user privacy.
While it is possible to obtain comprehensive user activity information,
e.g. user identity and key strokes from
surveillance cameras, 
we focus more on sensors that only provide
binary information like the presence of a user or keyboard activity
or other coarse-grained indirect data.
However, in some applications, privacy may not be an issue and part of
the security policy, e.g. access control policies, and we will also
present such applications.

Another source of environment information could be from physical
security information management systems.
Such systems are already in-place in many organizations and could provide
relevant environment information whereby user activities can be
derived, e.g. the door entry control system gives
evidence that a particular user is in the machine room.
Green buildings may have many sensors which are designed to detect human
presence or activity.

A natural application of our framework is to enhance the security of
intrusion detection systems (IDS).
This is because the IDS may be running on the host, 
as such, it could be compromised by malware.
External sensors, on the other hand, are
difficult to be accessed or compromised by either malware or remote intruders.\footnote{
Our focus is on remote attacks rather than attacks with physical system access.
} 
The results of an environment sensing based malware detector 
can be correlated with alerts from an IDS to reduce false positives.
While existing IDS may incorporate network traffic information in
gateways which are external to the host, the difference is that we
make further use of other tamperproof sensors and fuse it with user
presence and activity. 

The following example shows the difference
from the network intrusion problem. Consider the case of a single
email being sent. At the network level, there is insufficient
information to be able to identify whether it is a spam email
generated by malware or a legitimate email sent by the user. 
Whereas, in our framework, suppose that the
email is sent in the absence of the user and user activity, it is
reasonable to infer that the email has been sent automatically.
Furthermore, in the absence of any additional information, 
a default rule to classify this email as spam activity will be reasonably
effective and accurate.
In this section,
we present some application scenarios of malware detection where a botnet
is using malware on the host for:
(i) sending spam email;
(ii) contributing to distributed denial of service (DDoS) attacks; 
and (iii) using the host as a compute engine for offline
dictionary attacks.

A different set of security applications is where the
environment information is used on the host as part of a security mechanism.
It can be used to increase the expressivity of
access control and rate control policies
by requiring privileged actions on a host to be correlated with
physical user presence and physical activity.
This prevents remote attacks which
escalate privileges, e.g.
privileged actions can only
be performed if the administrator is present
and using the console.

An important property of our framework is that it can be immune
to attacks which compromise the host.
Without the fusion of sensor data from the environment surrounding the host, 
the attacker can simply hide inside the host or erase traces of an
attack or intrusion. Some malware may even be able
to shut down IDSs deployed in the host. A limitation of our
framework is that the sensor data obtained is coarse grained and
possibly noisy. Nevertheless, our experimental evaluations show 
such coarse data is already useful in identifying certain 
mismatches between the host's and user's physical activities.

The rest of this section is organized as follows. Sec.
\ref{sec:sensor-framework} introduces the framework of integrating the data
from external sources to the access control or intrusion detection
logic.
We apply the framework to malware detection in Sec.
\ref{sec:app-malware} to demonstrate that information
external to the host enhances the detection of malware activity.
The framework is extended to rate and access control applications in
Sec. \ref{sec:app-ac-rate}.
Related work is discussed in Sec. \ref{sec:sensor-related}
and Sec. \ref{sec:sensor-conclusion} concludes.
