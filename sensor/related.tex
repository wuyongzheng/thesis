\section{Related Work}
\label{sec:sensor-related}

If we consider a computer host to include the host, the user channel
and the network channel, then host security can be divided into:
({\it i}) software security, which ensures the software running in
the host is authentic, e.g. antivirus \cite{post1998use}, system call
filtering \cite{provos2003improving} and binary authentication \cite{halim2008lightweight};
({\it ii}) user security, which ensures the user is authentic, e.g.
password/biometric authentication, physical perimeters and
surveillance camera monitoring; and ({\it iii}) network security,
which ensures the network communication is authentic, e.g. personal
firewalls \cite{ingham2002history}. Our approach to enhance host
security is substantially different from the existing designs in
three aspects. Firstly, the model we propose fuses data from a few
channels of external environment sensors to monitor the host
activity. Secondly, as our model does not require controlling or
modifying the host operating system or software, it is able to provide some
level of security even when the host is compromised, rather than no security. 
Thirdly, our system
detects outbound or on-host malware execution, which complements
intrusion detection.

There are a few works which correlate information from different
channels to improve host security. BINDER \cite{cui2005design}
correlates user events (user input), process events (process
creation and process termination), and network events (connection
request, data arrival and domain name lookup) to detect malware.
Both our malware detection system and BINDER correlate user presence
information and system behaviour to detect malware. BINDER uses
information collected by the user's machine, which potentially could
be manipulated by the compromised host.

The systems proposed by Gu et al. \cite{gu2007bothunter} and Yen et al.
\cite{yen2008traffic} detect botnets by analyzing and correlating network
traces. The two systems and our malware detection system use
information from the network router rather than the host in question
so as to be able to deal with the case when the host is compromised
and gives false information. The difference with our malware
detection system is that we have user presence and activity
information in addition to network information.

Kumar et al. proposed a system 
\cite{kumar2005using,yap2008physical,kwang2009usability} which continuously monitors
user's biometric identity and locks up the computer if it cannot detect
the correct user.
Both their system and our access control system use physical user
information to provide additional factor for authentication.
There are two main differences between the two.
Firstly, our system only detects user presence information, while
their system detects user's biometric information which is much
stronger but gives less privacy.
Secondly, their system runs entirely in the user's machine, thus
it cannot guarantee the authentication once the machine has been compromised.

There are many works on location-based access control (LBAC) 
wireless networks, e.g. \cite{ardagna2006supporting}. 
LBAC models assume that the user devices are mobile,
their locations are tracked for service continuity, or verified
before granting access.  The problem we address is different in that
we are not focused on mobile devices, instead our focus is on
utilizing a combination of environment data to enhance security,
such as to detect malware on the host and to regulate the usage of
resources by the host.

There are also theoretical models which fuse information from multiple channels
for intrusion detection.
Siraj et al. proposed a method \cite{siraj2004intrusion} to fuse information
using artificial intelligence techniques.
Thomas et al. proposed a probabilistic model \cite{thomas2009improvement} to address
this problem.
Our work, on the other hand, addresses the physical sensor framework.
In terms of information fusion, we have employed
a simple rule based method, however, other methods could be adopted as well.
