\subsection{Handling Exceptions}

It is important to also deal with cases where there is exceptional
behavior which may otherwise cause false positives.

\subsubsection{Scheduled Outgoing Email}
If user has a large amount of email to send, and he opts to send them while going for a coffee break, our system supports that if the user has obtained sufficient email tokens. 
%for him to send the emails when he is away. 
The tokens to send email are obtained by directly interacting with the monitor,
such as by solving a CAPTCHA \cite{von2003captcha} or graphical Turing test
to validate that it is a human request.
Several tokens can be obtained in a batch, and the user only needs to solve one puzzle for validation.

Such a mechanism employing tokens to send emails can be adopted even when 
the user is present, as a special way of requesting for additional email quota. The tokens obtained for sending email when the user is present expire when 
the user goes away. 
Similarly, the tokens allowing the user to send email while he is away expire when the user comes back. The expiration mechanism reduces the opportunity 
for spam worms to hijack leftover tokens.

\subsubsection{Scheduled CPU Intensive Programs}
We need to handle carefully the case of non-malicious programs that commonly
run when no user is present. Besides system auto-updates, these
programs include nightly backup, nightly virus scan, remote desktop
and user scheduled computation. Backup and virus scan are regular
tasks. Their resource consumption can be profiled, in terms of the
start time, duration, and the increase in CPU temperature. Based on
our measurement, system backup only increases the CPU temperature by
1-3$^\circ$. Virus scan usually increases the CPU temperature by
2-3$^\circ$. These host profiles are basis of a whitelist kept by
the monitor to suppress false alarms. Even if attacker tries to hide
the malware execution by running it at the same time as these tasks,
the excess increase in temperature or duration will still trigger
the alarm.

User scheduled computations may cause irregular behavior.
As the execution of scheduled computation can be planned ahead, the
user simply declares in advance to the monitor, the estimated process
start time, duration and CPU load
(converted to temperature increases by the monitor).
The monitor suppresses any
false alarm within the specified resource consumption. The user is
advised to give a conservative estimation, for it is better for the
user to intervene if excess resources are consumed, than allowing
the attacker to free ride. 
