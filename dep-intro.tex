\subsection{Introduction}


Software is often made up of
a mix of different modules; from system components, to
software from various third parties.
One's software is then part of an eco-system of software modules
which forms a rich and complex web of relationships. 
System components (usually in the form of dynamic link libraries)
are also crucial and are usually needed for
a program to function.


Microsoft Windows has a very complex software module eco-system.
In addition, the usage dependency between
APIs with the actual software components delivered as binaries
is usually not properly documented. 
Microsoft software often makes extensive
use of Microsoft software components without much documentation.
While source code may be available, it would not be available for all
the relevant software modules.
Some modules are so extensively embedded in the system that they are
hard to study or understand.

This ``web of dependencies'' between software, 
means that
there can be fragile and implicit dependencies between all the software
installed (past, present or future) on a system.
In Windows, this often leads to software failures when installing or 
uninstalling some application.
This is colloquially known as ``DLL hell'' \cite{dllhell}.

In this paper, our objective is to understand the usage and dependencies
between software modules (executables and DLLs) in Windows.
We work purely with binaries since source is not available for all the
components. Furthermore, some software may be written in more than
one language.
Even when there is source, it will still interact in possibly complex
ways with other modules which are only in binary form.
We propose two visualizations: EXE dependency graphs and DLL dependency
graphs. The first is useful for comparing several programs
against each other, possibly this may include the involvement
of other processes not from those executables.
The second is useful for investigating the interactions between 
an executable and the DLLs used, as well as, interactions
between the DLLs themselves.

We deal with potential complexity of the visualizations in a number of ways.
The visualizations are
simplified either through a compressed dependency graph
or through the simple notion of grouping by function
or vendor (but other mappings are possible). 
Projection gives a simpler graph giving the interactions which
only affect selected modules; while the diff operation
gives the difference between two dependency graphs.

It turns out that both visualizations are simple, yet effective for
achieving comprehension of module interactions and dependencies
of realistic software on Windows. For example, we give scenarios
involving whole system usage from boot to shutdown, understanding strange
interactions which can occur when software is installed (TortoiseSVN modules
affecting arbitrary software), sharing of modules between different browsers,
understanding module usage in terms of source of the module 
(vendor perspective), 
visualizing usage of different versions of modules for networking, etc.
We believe that it is a good idea for software developers (and even users)
to understand the interaction between modules on a system.
This is useful for developing and maintaining 
robust software. Furthermore, the complexity of the
Windows environment is such that many unexpected interactions
and dependencies can occur without the knowledge of software developers,
administrators and users.

\subsection{Related Work}

Most of the related work on software dependencies makes use of analysis
of the source code. Some of it also makes use of static and 
dynamic analysis of binaries.
\cite{metrics} focuses on software testing and analysis of
software components and their dependencies on Windows Server 2003
using a tool from \cite{testing}.
\cite{network} is also similar and uses graph network analysis 
to predict software defect rates on Windows Server 2003.
Other tools to construct dependency graphs are \cite{progdep} and \cite{depvwr}
but they rely on the source code.

\cite{scenario} and \cite{hierarchy} use dynamic analysis to extract
views of software at different levels. The purpose of their tool is more likely
suitable for developers since their use-cases are tailored for helping
them locate areas to be changed. Their method relies on access to the source
code since they rely on instrumentation by adding code to intercept function
entry and exit points.

Our paper focuses on understanding interactions of software modules,
in terms of the binaries, as a whole in Windows through 
various dependency graphs. We deal with binaries and not with source code.
We also deal with the more complex environment of Windows, e.g. kernel
can call user space functions, a form of callback.


