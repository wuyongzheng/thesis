\subsection{Conclusion}
\label{sec:conclusion}

The objective of this work is twofold.
Firstly, we show that visualization can be beneficial for problems
in software understanding and diagnosis. We demonstrate this
for traces with a single program as well as with
several programs and a complete system trace.
Secondly, we show that a DotPlot-based visualization, \VDP{}, is effective
in visualizing a range of problems.
One feature which distinguishes \lviz{} is that as we work
at the operating system level with system and stack traces, 
we do not need source code of the software.
We believe that the range of understanding and diagnosis applications
shows that visualization of operating system traces is an interesting approach.
It also complements other kinds of analysis such
as program analysis and data mining.

Our \VDP{} visualization is only one general purpose visualization.
There are some other problems where other visualizations \cite{dep-icse} or special
variants of \VDP{} would be more appropriate. For example, to understand
resource usage, a special kind of \VDP{} with resources and traces overlaid
with resource lifetimes would give more information than a regular \VDP{}.
We have not taken advantage of source code and this can further enhance
the visualization but it would need monitoring infrastructure which
can correlate execution with the source.
We also remark that although we do not make use of source code, we
see that visualizations using program points can be used to understand 
how the code works (at the native code level).

We have developed the \VDP{} tool for Windows because such visualizations
are more beneficial given the closed source nature and system complexity
of Windows. 
However, \VDP{} is not reliant on a particular monitoring 
infrastructure and the same visualization ideas can be applied
to other systems with a rich source of execution traces, e.g. Unix.
