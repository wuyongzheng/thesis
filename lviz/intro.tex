% \subsection{Introduction}

Software is increasingly complex with many interactions with other software
and the operating system.
The previous visualization shows the complexity in the
aspect of module dependency.
Other factors include persistent state interaction, inter-process communication,
networking, etc.
This complicates understanding software/component behavior and diagnosing
problems or performance.
As discussed in Section~\ref{sec:bg-win},
the close source nature of Microsoft Windows (or simply, Windows)
exemplifies this trend.
Suppose our task is to understand/diagnose/debug some software on Windows.
Ideally, system/software documentation or source code
analysis would give the answer.
In practice, however, this may not be
workable either due to lack of source or overall
system complexity being too high.
An orthogonal approach to address this issue is by examining
detailed system behavior.
Recently, there are a number of monitoring systems which
give detailed system traces such as DTrace~\cite{cantrill2004dynamic} for Solaris,
DProbes for Linux and Flight Data Recorder~\cite{verbowski6flight}
and WinResMon (Sec.\ref{sec:winresmon}) for Windows.
However, as shown earlier in this chapter,
the system traces can be very large and also
difficult to analyze.

We propose \code{lviz}, a visualization tool for
understanding such large and complex system traces.
\code{lviz} has a number of visualizations.
In the thesis, we focus on the visualization which we call VDP.
It consists of a number of inter-connected sub-visualizations:
an extended DotPlot;
two Axis Histograms; and one or more barcodes.
The VDP can be flexibly configured for different purpose.
In Sec.~\ref{sec:lviz-vis}, we show the design and configurations of VDP.
In Sec.~\ref{sec:lviz-imp}, we show that
the prototype of \code{lviz} is efficient and can handle large system traces
with real-time response. For example, a trace of size 120MB with 558K events 
is loaded in under 3 seconds
and after that it can be interactively displayed/zoomed in around 0.5 seconds.
The VDP can be used on different traces and for different purpose
by using different configurations.
In Sec.~\ref{sec:lviz-study}, we use some case studies to show that
VDP can be used for solving performance problems,
program failure diagnosis and finding execution patterns/anomalies.
