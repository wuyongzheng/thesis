\subsection{Related Work}
\label{sec:lviz-related}

% \TODO{maybe dont need the first para}
% There are a number of related works on execution trace visualization,
% some are graph based, some are hierarchical,
% some are Cartesian coordinate based.
% We focus on Cartesian coordinate base visualization to which VDP belongs.
% The Cartesian coordinate based visualization is commonly used when each
% visualized entity (function, event, object, line of source code, etc.)
% is associated with some numerical properties.
% Each entity is then positioned on to the Cartesian coordinate based on
% its properties.

% \TODO{reviewer 2:
% The statement in section 2 that Cartesian visualizations are commonly used
% when each visualized entity is associated with two or three numerical
% properties is somewhat confusing. The choice of the so-called embedding, or
% projection, is in Infovis in general not one-to-one to the data attributes,
% at least not at this level. For example one would visualize some data using
% hierarchies primarily and not Cartesian plots if the most important aspect to
% reason about is the hierarchical one, even when the data has 2 or 3 numerical
% attributes per element. Or a table in a data base can be shown as a table,
% but better as a graph, if it actually encodes a graph. I think strongly that
% the layout choice should reflect the task at hand, and not data modeling
% issues.  fixed.
% }

% \TODO{reviewer 2:
% In the previous work you should definitely discuss the more recent work of
% Holten, Cornelissen, van Deursen, and Zaidman on visualizing trace
% information, see "Understanding Execution Traces Using Massive Sequence and
% Circular Bundle Views" (proc. ICPC 2007), since it is at least as related to
% the topic of this paper as the other references, and is quite recent. The
% work of F. van Ham and J. Abello on MatrixZoom fits also in this area.  fixed
% (only added the first paper).
% }

SeeSoft~\cite{eick1994graphical} visualizes general log files by zooming out
lines of text logs and coloring them by message types.
SeeLog~\cite{eick1996displaying} visualizes Unix accounting trace files by
plotting events by time and process.
SeeLog can be viewed as a restricted variant of the
VDP visualization.
Magpie \cite{barham2004using} is a performance analysis tool designed for web servers.
It uses collected system events including file I/O, computation,
thread scheduling and network to analyze the resources usage of
HTTP requests.
BootVis~\cite{bootvis} 
is a visualization tool to tune
Windows booting plotting resource use versus time.

Bodic et al. \cite{bodik2005combining} propose another visualization tool for
web servers.
It shows frequency of different HTTP requests
and highlight anomalies that may indicate site failures.
VDP on the other hand is a general trace visualizer but
it can be used for similar diagnosis (see Sec.~\ref{sec:wbbench}).
TuningFork \cite{bacon2007tuningfork} is a trace analysis tool for real-time systems
including visualizing unsatisfied real time constraints in
real-time systems.

EVolve~\cite{wang2003evolve} is a visualization framework for understanding
Java program by visualization execution traces from the JVM and
has a DotPlot visualization.
% It includes a DotPlot module which has some similarity with VDP.
However, VDP uses an extended DotPlot (see Sec.~\ref{sec:lviz-vis})
and generates different visualizations (see Fig.~\ref{fig:make-matching})
depending on its configuration.
\code{lviz} is designed to scale for large traces and
to be efficient to meet user interaction needs.
Voigt et al.~\cite{voigt2009object} propose a visualization method which correlates
class method invocation, object invocation and time.
To correlate methods and objects, it places objects and methods
on each axis and draw dots to represent the
method-access-object relationship.
Cornelissen et al.~\cite{cornelissen2007understanding} propose two visualizations,
massive sequence and circular bundle views, to examine execution traces.
The massive sequence view visualizes a trace by placing each event as rectangles
according to its invoking software module (X-axis) and time (Y-axis).
The circular bundle view visualizes software module interactions by arranging
modules into a tree and draw two-module interaction as a curve along the
path in the tree.
These visualizations are different as they do not compare events 
nor work with closed source software.

% \TODO{reviewer 3:
% Dotplots and their extensions have been studied in the field of visualization
% more than extensively. They have been proven to be useful for finding
% patterns in similarities in various fields. In this paper, the dotplot is not
% used differently. The configuration of visualizations reminds me DNAVis: Case
% Study: Visualization of annotated DNA sequences, Tim Peeters, Mark Fiers,
% Huub van de Wetering, Jan-Peter Nap, Jarke J. van Wijk, Joint Eurographics
% IEEE TCVG Symposium on Visualization (2004).
% }
