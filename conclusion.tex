\chapter{Conclusion}
\label{sec:concl}

This chapter concludes this thesis.
We summarize the contributions of this thesis in Section~\ref{sec:concl-summary},
and discuss some future directions in Section~\ref{sec:future}.

\section{Summary of the Thesis}
\label{sec:concl-summary}

In this thesis, we propose our monitoring frameworks: LBox and WinResMon.
The information collected by them can be used to solve many problems including
software comprehension, fault diagnosis and system security.
Based on the frameworks, we developed two visualizations: module dependency
visualization and \code{lviz}.
As the frameworks are resource-based, we can easily adapt them to control
the access of resources in system and apply them on system security.
Our binary integrity system prevents loading of untrusted binary by monitoring
file access in the operating system.
To detect malware when the operating system kernel is possibly compromised,
we correlate information gathered from external sensors and network routers.

\paragraph{LBox}
We show our user-level monitoring framework LBox.
User-level monitoring differs from traditional super user operated monitoring
in that user-level monitoring can be safely used by all users without
worrying about confidentiality and denial-of-service problems.
LBox also allows cascading of monitors.
Comparing state of the art monitoring frameworks such as Solaris DTrace
and SystemTap, LBox does not allow user supplied script to run in kernel
space.
This reduces the complexity and more importantly reduces the chances
of introducing security bugs of compromising the kernel.
To compensate the flexibility of in-kernel script, LBox provides
a fine-grained event filtering API.
Our experiments are very encouraging
showing that the overhead is comparable to in-kernel
mechanisms such as DTrace.

\paragraph{WinResMon}
\TODO{it's same as winresmon's conclusion.}
We presented the motivation, design, implementation and usage of
WinResMon.  Its main use is to inspect resource access and software dependency
issues in Microsoft Windows environments.  As WinResMon is extensible, system
administrators can also build tools using WinResMon for custom queries and
system analysis.  Benchmarking shows that WinResMon is reliable and is
comparable to other popular tools.

\paragraph{External Monitoring}
When the operating system kernel is compromised, information collected
from the host cannot be trusted.
We propose a framework that incorporates external information from sensors in
securing host computers.
Since the sensors are external, they are difficult to be
accessed and tampered by a compromised host.
The framework can be applied to detect malware and also mitigate the
impact of a compromised host.
Our experiments show that our framework can successfully detect
three types of malware: email spammer, DDoS zombie and password cracking
zombie.
The framework also takes advantage of the growing popularity of
pervasive computing and sensor networks which make it
feasible for cost-effective deployment in the near future.

\paragraph{Module Dependency Visualization}
\TODO{same as the section's conclusion.}
We have demonstrated that software dependencies in Windows are quite
complex even when looking at the coarse grain level of software components
packaged into binaries and executables.
Our module dependency visualization gives an effective way of extracting and
visualizing the software dependencies and interactions between binaries.
We show that even with the complexity of actual Windows software,
it is possible to analyse whole system interaction,
understand how modules are used and shared,
and also discover potential unexpected or unusual interactions/modules.
Such an understanding is also useful for software developers,
system administrators and also users, to manage the
software ecosystem on Windows and to deal with the problems which arise
from module updates and potential ``DLL hell'' repercussions.

\paragraph{LViz}
We present a DotPlot-based visualization for studying execution traces.
We use examples to show that LViz is flexible to look at a wide range
of problems.
As the traces can be very large, LViz is implemented to be efficient
and responsive for interactive use.
We have developed the VDP tool for Windows because such visualizations
are more beneficial given the closed source nature and system complexity
of Windows.
However, VDP is not reliant on a particular monitoring
infrastructure and the same visualization ideas can be applied
to other systems with a rich source of execution traces, e.g. Unix.

\paragraph{BinAuth}
We performed a study on the feasibility of mandatory
software component authentication. Our approach uses a lightweight
authentication technique using message authentication code.
According to our benchmarks, we
found that using a cache reduces the overhead as it reduces
the number of repeated authentication requests.
Based on these benchmarks,
we can conclude that our mandatory authentication can be added to Windows
without significant overhead.
We have also introduced the concept of ``software naming''. The
idea behind this is to uniquely identify a file. We have mentioned a few
applications for this idea.
Given the security benefits of binary authentication and software naming,
and only small overhead incurred even on complex OS like Windows,
we envisage that the presented results can better convince
the OS and user community alike to start deploying them more universally
in realizing secure software distribution and execution.

\paragraph{BinInt}
Based on BinAuth, we propose BinInt, a security model which caters to the
dynamic use of binaries within the software life cycle
while protecting against attacks in default mode
and giving isolation between software domains in install mode.
Our prototype is efficient and usable while
protecting a broad range of binary loading/execution mechanisms in Windows.
We found our system to be mostly transparent in usage on typical
Windows software throughout its software lifecycle.
Thus, BinInt is a practical solution
which gives a good tradeoff between usability and security
to protect binaries on Windows.
Our model can also be combined with other security mechanisms.

\section{Future Work}
\label{sec:future}

In future work of monitoring infrastructures include functional
extensibility and performance optimization.
More types of events such as internal state changes and low level
hardware operations (hard drive, keyboard and mouse) can be included.
As multi-core CPUs becoming more common, we can optimize the
implementation to make use of them.
We can apply our framework in the cluster or cloud computing setting,
and have query API to aggregate information from multiple hosts.
Log compression can benefit from the high redundancy of the logs.
We can have an operating system independent
abstraction which make it easier to port for future operating systems.

Our binary integrity model, BinInt, focuses on binaries only.
However, securing only binaries may not be sufficient.
For example, the attacker can modify Java class files to
change the behaviour of java programs.
The same applies to shell scripts and configuration files.
We can generalize BinInt to protect the integrity of any kind of file.
The difficulty is that data files are much more dynamic than binaries.
Unlike binaries which are only changed during install, update and removal
of the software, data files are usually changed much more frequently.
This makes direct use of d-mode for data files unusable.
More fine grained policies can be applied for data files.
