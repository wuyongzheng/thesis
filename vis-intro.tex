%\section{Introduction}

\TODO{expand}

In this chapter, we show two trace visualizations which make use
of the system and software traces generated by our monitoring infrastructure.
\TODO{check if it's defined earlier.}
We use the term system trace to denote the trace of the interaction
among the OS and different software.
We use the term software trace to denote the execution trace of a single
software, whose level of detail can be system call only as in LBox
(\autoref{sec:lbox}), a filtered system trace, or
instruction trace (\autoref{sec:instrumentation}).

Section \ref{sec:depvis} shows our first visualization, which
uses system traces to discover inter-software
dependencies; and uses software traces to
discover inter-component dependencies.
Section \ref{sec:lviz} shows our second visualization, which
is more general and can be used for different
purposes including software failure diagnostics, analysing performance
issues, anomaly discovery, etc.
Although both visualizations are designed to use traces from any OS in general,
we use Windows traces as illustration because
the closed source nature and the large and complex software eco-system make
software understanding very difficult and important.
