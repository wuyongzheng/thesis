\rule{\textwidth}{1pt}
{\em
\TODO{...}
}\\
\rule{\textwidth}{1pt}
\medskip

Securing Windows is a challenge because of its large
attack surface which can lead to many ways where
binaries (Sec.~\ref{sec:wi-binary}) can be loaded and subsequently executed.
Many of the system security problems such as malware
stem from the fact that untrusted binaries are executed.
This is further complicated by
software in Windows having many interactions on binaries such as
automatic online updates, installing third party plugins,
hijacking system libraries and creating temporary binaries.
These interactions are often not well understood or specified.
With the help of our monitoring infrastructure (Sec.~\ref{sec:winresmon})
on the interactions with binaries,
we can tackle the security problems from the {\em binary integrity}
point of view.
Binaries are the only type of files that contain native code which
can be directly loaded by the operating system.
Many security attacks such as thumb drive propagated virus and
drive-by downloads are directly targeted on binaries.
Most other attack methods such as buffer overflows and social engineerings
are usually performed together with introducing malicious binaries
so that the attacks can be persistent to the system.
The term binary integrity means binaries in the system are
wanted and trusted by the user.
% Since our monitoring infrastructure tracks creating, modification
% and loading of binaries, we can protect binary integrity
In this chapter, we present our security system which ensures
binary integrity in Windows.

The security system has two goals. Firstly, the security objective
is to ensure the use of trusted binaries and maintain their integrity.
This protects against attacks which attempt to modify or remove binaries
commonly used in malware or denial of service attacks.
We prevent malware attacks which rely on
getting a malicious executable to be run or
a malicious DLL to load.
In addition, we prevent the malware from being persistent.

Secondly, the security mechanism needs to be usable with typical
software in binary form.
It needs to be compatible with the software life cycle of software in the
system, i.e.
software may be installed, then used, then updated, and 
finally uninstalled. 
Portions of installed software, e.g. DLLs,
may be shared with other software. 
Ideally, the security mechanism should be (mostly) transparent to 
this software life cycle.
While it may not be possible to be completely transparent/compatible 
for arbitrary binaries, it should support the mechanisms used 
by typical software.
We remark that as we are concerned with
closed source and binaries,
approaches which change the process of installation/update,
usually requiring (some source code), are not relevant.
Clearly, there is a tradeoff between security and usability --
we seek a practical middle ground with improved security while
being compatible with typical software.

This chapter is organized as follows.
In Section~\ref{sec:binauth},
we first present {\em BinAuth} which achieves the first goal.
BinAuth can be used in a relatively static environment where the binaries 
rarely change or binary changes are well managed by the system administrator.
The aim is to have a reliable and efficient security mechanism
to authenticate binaries.
This joint work has been published in \cite{halim2008lightweight,wu2009esi}.
The author of this thesis focuses mainly on designing, implementing and testing
the kernel mechanism and the user space signing tools.
The Software ID scheme and the hash-based message authentication code
is designed by other team members.
We then extend it in Section~\ref{sec:binint} to make it more usable
in a general non-static setting.
Our new system {\em BinInt} allows easy installing and updating software
while still prevent untrusted software from running.
The work has been published in \cite{wu2011towards}.
