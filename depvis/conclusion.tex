\subsection{Discussion}

In addition to the scenarios in Sec.~\ref{sec:apply},
we experimented with other scenarios which we describe briefly due to lack
of space. They are similar to those in Sec.~\ref{sec:apply} but the details differ.
We investigated what happens with Internet Explorer under 
the following situations:
(i) loading a Java applet versus an {\tt about:blank} page; 
(ii) a demonstration Microsoft Silverlight webpage versus blank; and
(iii) a {\tt https} webpage versus regular {\tt http}.
Another scenario is to answer the question of
what are the new dependencies for Acrobat on a regular PDF versus
one with an embedded form.
We also experimented with Java applications and .NET applications 
to see the behavior of virtual machines and managed code.

The two visualizations EXE dependency and DLL dependency graphs are just two
ways to visualize software module dependency using execution traces.
We can have other different visualizations using the traces.
For example, instead of using binaries as basic units, we can use functions
to visualize more fine grain behaviour.
This makes the graph much more complex because there are many more functions
than binaries.
We can use similarity grouping, diff and projection to ``zoom in'' 
on specific regions.
Because we utilize instrumentation of binaries, we can also visualize other runtime
properties of the code.
For example, visualize the CPU time used by different modules,
and scale the nodes in the graph so that the area of each node is proportional to the time
(or instructions) spent in the binaries represented by the node.

Although we have focused on visualizations in this paper, the traces which are collected
contain a wealth of information. A workflow for software comprehension can also make use
of the trace data in addition to the visualization. 
The trace data is at a fine grain level but is hard to understand. 
Thus, the visualization gives a roadmap for investigating the traces.

We have demonstrated that software dependencies in Windows are quite
complex even when looking at the coarse grain level of software components 
packaged into binaries and executables.
Our tools give an effective way of extracting and
visualizing the software dependencies and interactions between binaries.
We show that even with the complexity of actual Windows software,
it is possible to analyse whole system interaction, 
understand how modules are used and shared,
and also discover potential unexpected or unusual interactions/modules.
Such an understanding is also useful for software developers,
system administrators and also users, to manage the 
software ecosystem on Windows and to deal with the problems which arise 
from module updates and potential ``DLL hell'' repercussions.

\subsection{Acknowledgements}

We acknowledge the support of the Defence Science \& Technology Agency.

