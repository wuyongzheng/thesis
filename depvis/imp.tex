\section{Implementation}

Implementation of the visualizations entails collecting tra\-ces when
binaries are loaded and tracking of function call and return.
Binary loading information is obtained using
a modified version of WinResMon~\cite{winresmon},
which makes use of a kernel driver to obtain
what binaries are loaded by a process as it runs and
what child processes are created.
The EXE dependency graph is created from
the WinResMon trace file into a dependency graph in
Graphviz \cite{graphviz} format and rendered by {\tt dot} in Graphviz.

The system trace from WinResMon is very robust and allows for continuous
full system tracing if so desired.
It can also trace a large portion of the system boot,
see Fig. \ref{sec:apply:boot}.
It also works with software which employs self-modifying
code and other malware techniques to defend against virtual machines,
debuggers and code instrumentation such as Skype.
Thus, it can handle many more cases than with a more detailed
instruction instrumentation or interpretation approach, as is done
with PIN below.

The DLL dependency graph is built from a runtime contr\-ol flow trace from
function calls and returns during execution.
We employ PIN~\cite{pin} which instruments the binary to allow monitoring
of the control flow during execution of call and return instructions
when a process runs saving these events into a trace file.
The trace is then analyzed to generate a DLL dependency graph in
GraphViz format.

The DLL dependency graph only needs instrumentation of the call instructions
which is straightforward.
DLL projection, however, requires tracking of the control flow of both
call and return instructions.
The full call/return tracking is more complex than one would expect
under Windows because one has to handle:
non-local goto's from {\tt setjmp()} and {\tt longjmp()}, exception handling,
kernel callbacks and multithreading.
For example, a naive implementation which matches counts of call and
return instructions will fail when a {\tt longjmp()} is used
since that means that some function does not return.
We make use of stack and return address matching to do this
tracking, which is not described in detail for lack of space.

In Windows, execution can be interrupted to switch to a new context.
When execution in the new context completes, execution of the original
context resumes.
There are a number of reasons why the context changes
such as: asynchronous procedure call (APC), kernel callbacks
(kernel calls a user function) and structured exception handling.
Thus, context switching has to be tracked to avoid matching
call and return from different contexts.
We remark that the start of a context switch can be obtained in PIN but
not when the context terminates.
Instead, termination is observed by watching the {\tt NtCallbackReturn}
and {\tt NtContinue} system calls.
Context switching can be cascaded, thus, we maintain the unfinished
contexts in a stack.

We generate a trace file to record the call/return control flow.
The trace records events such as function calls/returns,
thread creation/termination, context switching, binary loading
and system calls.
This allows us to analyze the execution trace in many ways to generate
different graphs.

The DLL initialization in Windows is identified by the special
DLL entry function {\tt DllMain} which is called when a DLL is loaded.
To remove the control flow effect due to DLL initialization,
we remove all calls which are due to execution of {\tt DllMain}.
