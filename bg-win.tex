\section{Windows Issues}
\label{sec:bg-win}

The Windows NT operating system is rather complex and different from other
operating systems.
It has many unique features and mechanisms which impact on understanding,
monitoring and security.
We now discuss some of the these which are related to the thesis.

\subsection{Closed Source}

The Windows operating system is a closed source system.
Firstly, the kernel is closed source.
This makes kernel monitoring very difficult.
Dynamic instrumentation tools like DTrace~\cite{cantrill2004dynamic} and
SystemTap~\cite{prasad2005systemtap} are not relevant because their probes are
specific to code points or functions in the kernel.
Without understanding the purpose of each function, probes are meaningless.
It makes kernel extension difficult as well.
The lack of kernel APIs make anti virus developers use undocumented internal
functions in a hacking way.
For example, the Kaspersky is known \cite{skywing2006avwrong} to patch internal
kernel functions, which makes it only work on 32-bit but not 64-bit systems.
In our WinResMon (Sec~\ref{sec:winresmon}) work, we monitor system calls
by hooking the kernel dispatch table, which is a well-known
system call monitoring technique,
but not officially supported by Microsoft and may cease to work
after a Windows update.
Unfortunately, there is no officially supported technique to achieve this.

Secondly, the semantic of system calls is closed.
Unlike UNIX, programs do not directly invoke systems in Windows.
They call higher level APIs, which may call some other APIs, which make
the system call.
The association between higher level APIs and the systems is complex and
again closed.
For example, to open a file in UNIX, one may call the \code{open(2)}
system call directly.
In Windows, one should call the officially documented API \code{CreateFile()}.
\code{CreateFile()} calls \code{CreateFileA()} which calls
the system call \code{ZwCreateFile()}.
One may think this is not a problem because we can just monitor the documented
API layer and ignore the system call layer.
However, not all programs follow the documented API.
To make reliable monitoring, system call have to be monitored.

Thirdly, the interaction among the components is closed.
Windows has microkernel operating system features which make
some tasks, such as networking, printing and graphical interface be partially
handled by user space services.
In other words, a process can perform tasks on behave of another process.
This feature can be exploited to circumvent monitoring or
security mechanisms.

\subsection{Super User Account}

The early versions of Windows (Windows 95, Windows 95 and Windows Me)
are single user operating systems, thus do not distinguish normal and
super user accounts.
Windows NT introduced the multiple user operating system, which
separates user configurations and introduces normal/super user account.
The super user account (also known as administrator) has higher privilege and
is supposed to only perform administrative tasks following the
least privilege principle~\cite{saltzer1975protection}.
However, in practice,
most users choose to use the super user account,
because some software written for older Windows do not work if running using
normal account.
Furthermore, the first account created during Windows 2000 and XP installation
is by default administrator,
thus running normal account is an opt-in feature and
many users are even not aware of using administrator account.

In modern multi-user operating systems,
(i) separation of kernel and user context;
(ii) separation of different processes' address space; and
(iii) separation of different users' configurations are
very important concepts of security.
When programs run under the super user account,
all these separations are invalidated because the super user
is able to load kernel drivers, modify arbitrary process's state
and arbitrary files.
As a result, the recent versions Windows Vista and Windows 7 introduced
User Account Control (UAC) in order to mitigate the security problems of
super user account and promote the use of normal user account.
When a program running in super user account performs administrative
operations (listed below),
a UAC prompt is displayed and the user can choose to authorize or prevent
the operation.
It is designed to prevent malware from {\em automatically}
perform these operations.

\begin{itemize}
\item Installing and uninstalling applications
\item Installing device drivers
\item Installing ActiveX controls
\item Installing Windows Updates
\item Changing settings for Windows Firewall
\item Changing UAC settings
\item Configuring Windows Update
\item Adding or removing user accounts
\item Changing a user's account type
\item Configuring Parental Controls
\item Running Task Scheduler
\item Restoring backed-up system files
\item Viewing or changing another user's folders and files
\end{itemize}

There are a few problems with UAC.
Firstly, UAC only cares about administrative operations, which
are to do with system settings, but not user settings.
Thus, this is aimed at protecting the system but not the user, i.e.
malware which do not modify system resources are not affected.
This is alright from a multi-user security perspective, which focuses on
preventing a user from interfering other users.
However, in a single-user situation, which is mostly the case for Windows PC,
it is more relevant to prevent an application from interfering with other
applications of the same user.
For example, UAC cannot prevent malware from stealing web browser
cookies or modifying Word documents.
Some software, such as Google Chrome, are by default
installed in the user's home directory instead of the \path{Program Files}
directory and is consequently not covered.
UAC does not protect their binaries from being modified.
Secondly the protection from UAC may be illusory --- a common complaint is that
frequent UAC prompts leads to users
blindly allowing UAC queries \cite{motiee2010windows}.
Lastly, the UAC prompt does not give much information to make a decision,
which is essentially whether a particular executable is trusted for an
operation.
Most users (including technical ones) would not be able
to decide if the operation should be allowed.

\subsection{Software Management}
\label{sec:wi-management}

A software management system controls the installation, updating and
removal of software in the operating system.
Open source OS, such as Linux, commonly uses a package management system.
Examples are the Redhat Package Manager (RPM) for Redhat and Fedora Linux,
the Debian package management system for Debian and Ubuntu Linux.
Mobile operating systems such Apple's iOS and Google's Android have
similar application managers.

There is no consistent software management in Windows.
Most software products have their own installers, which perform installation in
different ways.
They have their updaters as well.
Some check online for update at each execution.
Some perform regular checks, which involves running
a service in the background.
Probably the only common thing is that they all register their software
removal programs so that the ``Add/Remove Program'' tool knows
how to remove them.

Without a consistent software management system, binaries in windows
is rather ``chaotic''.
Firstly, it is not possible to systematically tell which software a binary,
or file in general, belongs to.
There is no database to record this relationship as in RPM.
The directory structure is not reliable to figure this out because
binaries can be installed anywhere in general.
For example, some software install binaries in the \path{C:\windows\system32}
directory which is used to store core system files.
Even worse, a software installer can overwrite binaries installed by another
software.
Secondly, the software dependencies are unknown.
Package managers such as RPM keeps detailed and accurate
package dependencies and conflicts
so that when installing a software, it installs its dependencies as well.
Because Windows lacks this knowledge, software tends to bundle its dependencies.
This introduces a lot of duplicates in the system and makes software
update difficult.

Even with software management systems, the problems cannot be fully solved
because the systems do not provide mandatory protection.
They can only maintain the software installation under the assumptions that
(i) the software package is centrally created and signed by
a single distributer, e.g. RedHat;
(ii) software is installed properly using their tools; and
(iii) the installed files are not modified outside the
software management system.
However, any of the assumption can be false.
There is often some software that is not included by the distributor,
either because it is new or it does not satisfy the distributor's
requirement.
Third party software packages are then made.
Examples are the Google Chrome web browser for most Linux distributions
and the Cydia application repository for Apple's iOS.
These packages can conflict with existing or future first party packages.
Package signature can partially solve the problem, but users usually
ignore the signature.
The software management systems do not provide mandatory protection
mechanisms to prevent installed files from being modified.
Once the root privilege (or any software installation equivalent privilege)
is acquired, any file from any software can be modified.

\subsection{Binaries}
\label{sec:wi-binary}

Parts of the thesis, including module dependency (Section~\ref{sec:depvis})
and binary integrity (Chap.~\ref{sec:auth}) focus on binaries in Windows.
Throughout the thesis, we use the term binary to denote a file that contains
native executable code and can be directly loaded by the operating system kernel.
There are generally three types of binaries in modern operating systems.
An {\em executable} file is the main and firstly loaded binary of a process.
Executable files in Windows are conventionally (not necessarily)
given the \code{.exe} file extension.
% Executable files in UNIX variants are usually located in \path{/bin} or
% {\path/usr/bin}.
A process loads a single executable file.
A {\em dynamic linked library} (or DLL) contains native code which is designed
to be shared by different software.
A process can load many DLLs and a DLL can be loaded by many processes.
In Windows, DLLs are usually given the extension \code{.dll}, but other extensions
such as \code{.ocx}, \code{.cpl} and \code{.ime} exist as well.
A {\em kernel driver} can be loaded by the kernel and its code executes in kernel
space.
They usually have the \code{.sys} extension.
We list some of the common binaries and their conventional extensions below
(this list is not intended to be exhaustive):

\begin{itemize}
\item Applications (\code{.exe}) ---
the executable associated with a process
\item Command files (\code{.com}) ---
legacy executables for the MS-DOS environment
\item Dynamic linked libraries (\code{.dll}) ---
libraries loaded implicitly by dependency or
explicitly by \code{LoadLibrary} function at runtime
\item ActiveX controls (\code{.ocx}) ---
software components which
implement the Microsoft Component Object Model (COM). 
ActiveX controls are most commonly used by Internet Explorer.
Browser helper objects(BHO) are also
an example of ActiveX controls.
\item Device drivers (\code{.drv} and \code{.sys}) ---
kernel loadable modules
\item Screensavers (\code{.scr}) ---
executables used by Windows to display screensavers
\item Control Panel applets (\code{.cpl}) ---
applets for the Control Panel
\item Input Method Editors (\code{.ime}) ---
used by the Windows On-screen-keyboard to support different languages
\item Codecs (\code{.acm} and \code{.ax}) ---
bundled into Windows and also can be installed by 3rd party software.
The codecs are used for playing audio and video media.
\end{itemize}

\begin{figure}[htb]
\small
\begin{verbatim}
c:\windows\apppatch\acgenral.dll
c:\windows\system32\avgrsstx.dll
c:\windows\system32\imm32.dll
c:\windows\system32\lpk.dll
c:\windows\system32\msacm32.dll
c:\windows\system32\msctf.dll
c:\windows\system32\msctfime.ime
c:\windows\system32\shimeng.dll
c:\windows\system32\usp10.dll
c:\windows\system32\uxtheme.dll
c:\windows\system32\winmm.dll
c:\windows\system32\winspool.drv
c:\windows\winsxs\x86_microsoft.windows.common-controls_
6595b64144ccf1df_6.0.2600.5512_x-ww_35d4ce83\comctl32.dll
\end{verbatim}
\caption{Binaries loaded when running \code{notepad.exe} in Windows XP}
\label{fig:notepad-bin}
\end{figure}

We briefly discuss the binaries loaded when we run \code{notepad.exe},
the simplest text editor of Windows, as shown in Figure~\ref{fig:notepad-bin}.
Binaries with different extensions --- \code{.dll}, \code{.drv} and \code{.ime}
--- are loaded by the simple text editor.
We highlight that \code{avgrsstx.dll},
which is one of the DLLs of the anti-virus software AVG,
is ``injected'' into every processes.
This hacker-style DLL injection technique is not officially supported and
is also used by many malware.
This makes some anti-virus software behaves like malware.
The \code{comctl32.dll} with long pathname is version 6.0.2600.5512 of
the common control DLL, which keeps several different versions
for backward compatibility. 
The \code{winspool.drv} is the user space component of the printer
driver.

Loading of binaries is the most common way to lead code execution.
The complexity of binaries in Windows brings challenges to software understanding
and security.
We will see this in more detail in our visualization
(Chap.~\ref{sec:vis}) and binary authentication (Chap.~\ref{sec:auth}) work.

\subsection{Other Issues}

In Windows, the configuration is stored in a central database named the
{\em registry}.
This includes all kinds of configuration, such as
operating system settings like TCP buffer size;
per-user configurations like desktop background;
and software configurations like default font size
for Adobe Photoshop.
The centrally managed configuration database is a unique\footnote{
There are other central configuration systems such as
GConf for GNOME applications.
However, they are not as widely adopted as the registry in Windows.
} feature of Windows.
UNIX variants usually manage configurations in a per application basis.
Each application keeps its own configuration file (typically a
text file in \path{/etc/} or the home directory).
Our WinResMon (Section~\ref{sec:winresmon}) monitors the registry
related system calls.
It can monitor program accessing {\em individual} settings.
If a text based configuration file is used, this is much more difficult
and may require data flow analysis.
In this sense, WinResMon benefits from this feature.
In Section~\ref{sec:lviz-study}, we will use our LViz to study the
software behaviours of accessing configurations by visualizing the
log generated by WinResMon.

Windows has a different way in organizing the file system hierarchy.
While UNIX variants organize the file system in a single tree,
Windows adopts the notion of drive (or volume).
Each drive contains a separate file system tree, thus all drives
form a file system forest.
Historically, file and directory names are constrained by the 8.3
standard,
i.e. maximum eight letters name with three letters extension.
Later versions of Windows allow longer name, but for backward
compatibility, each long file name is automatically assigned by kernel a
8.3 name, so that it can be accessed by legacy software.
For example, \path{C:\Program Files} is equipment to \path{C:\PROGRA~1}.
This feature is often exploited to circumvent security systems with
file blacklist mechanisms.
In the kernel, files are named in a different way.
For example, the user space pathname \path{C:\foo.txt}
is equivalent to the kernel space pathname
\path{\Device\HarddiskVolume1\foo.txt},
depending on the underlying hard disk and partition layout.
Similar to most UNIX flavoured file systems, symbolic and hard links
are supported by the Windows NT file system (NTFS).
All these features make a challenging task to obtain unique identifiers to files.
This problem is answered in our binary authentication work (Section~\ref{sec:binauth}).

