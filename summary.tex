\chapter{Summary}

Operating system monitoring is an essential method of obtaining
information on running operating systems.
The information can be used to understand programs or
the operating system kernel.
It can be used to verify correctness of the execution or
discover problems such as performance bottlenecks and security flaws.
This thesis presents our monitoring infrastructures and uses them to solve
various problems on software comprehension, software diagnostics and
system security.

We first present two monitoring infrastructures, LBox and WinResMon.
LBox is a monitoring infrastructure on UNIX variants such as Linux.
It features novel user-level monitoring and recursive monitoring, which
make LBox safe to be used by unprivileged users in a multi-user environment.
It is light-weight as it can be implemented with very little kernel patching;
while its performance is comparable to state of the art monitoring systems
such as Solaris DTrace.
Our second infrastructure, WinResMon, monitors resource usage in Windows.
The closed source nature makes Windows internals obscure.
Traditional system call based monitoring would not make sense
because the semantics of system call names and parameters are not
generally understandable.
Resource-based monitoring, in contrast,
monitors software behaviour on its resource usages such as
file/registry, network and process/thread operations.
As an infrastructure, WinResMon supports APIs which
can be used to build tools for system administrators.
Our benchmarking shows that WinResMon is reliable and
is comparable to other popular tools.

Our two infrastructures are host-based, i.e.
the monitoring system and the monitored software run in the same host.
If the kernel of the host is compromised,
which is the case for Rootkit,
the information from the monitor cannot be trusted.
We propose external monitoring which obtains information from entities,
such as network routers and environment sensors, that are outside the host.
We use the sensors to monitor human user presence and correlate
this information with network traffic to detect malware in the host.
Moreover, we mitigate the impact of malware by limiting its resource usage,
which is done by adapting WinResMon from resource usage monitoring to
resource usage control.

With the large amount of information obtained by our system monitor,
we have developed techniques to visualize it.
We use system traces together with function call trace to visualize
software module dependencies.
As the number of modules can be very large,
we developed a number of ``zooming in'' techniques including
grouping of modules; filtering by causality;
and the ``diff'' of two dependencies.
Our second visualization, named \code{lviz}, discovers patterns and anomalies.
It is highly configurable to suit different purposes.
As shown in our case studies,
it can be used for software failure diagnostics,
analysing performance issues and other strange behaviours.

Many of the system security problems such as malware stem from the fact that
untrusted binaries are executed.
Since our WinResMon monitors file system related information flow,
we can tackle the binary trustworthiness from the information
flow point of view, similar to the Biba Integrity Model.
In short, low integrity process should not modify high integrity binary and
high integrity process should not load low integrity binary.
We achieve this goal in two steps.
We first implement a secure and efficient {\em binary authentication} system
which only allows binaries in a white-list to be loaded.
We then apply it on our {\em binary integrity} security model.
The security model prevents binary related attacks such as DLL planting,
drive-by downloading and phishing attacks;
while it is usable under typical usage scenarios including
software running, installation, updating and development.

Many parts of the thesis is implemented in Windows because of the great
varity of software and number of users which also attract many attacks.
The closed source nature also makes the monitoring challenging and
demanding.
However, the idea can be applied on other operating systems.
