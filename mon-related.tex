\section{Background and Related Work}

In this chapter, we give some background and related work on system monitoring.

\TODO{define system monitoring}

Related work can be classified in a number of ways.
From the enforcement point of view, we have discretionary and
mandatory monitoring.
Discretionary monitoring requires that the monitored software
actively report to its monitor.
The traditional UNIX \code{syslog} is an example of discretionary monitoring.
A log entry is generated when the monitored software calls \code{syslog(2)}.
The naive \code{printf()} debugging technique is also discretionary monitoring.
In contrast, mandatory monitoring systems enforce that logs entries are always
generated when certain actions are performed by the monitored software.
The \code{ptrace(2)} interface and Solaris Basic Security Module (BSM) Auditing
are examples of mandatory monitoring.
Mandatory monitoring is more suited for security purpose because of its enforcement.
Discretionary monitoring may give more friendly output since the monitored
software knows which pieces are more important.

A correlated classification is transparent/opaque monitoring.
In transparent monitoring, the monitored software does not need to be adapted
and sometimes is not aware of being monitored;
whereas in opaque monitoring, the monitored software need to be either
rewritten and recompiled or transformed manually.
The two types of classification are usually correlated because transparent monitoring
is usually mandatory as well, and opaque monitoring is usually discretionary.

 From execution environment point of view, the monitor can be executed
in a number of environments.
\begin{itemize}
\item The monitor can be executed in the process (i.e. same address)
of the monitored software.
The naive \code{printf()} debugging technique belong to this category.
A common technique to monitor API or DLL function calls is to rewrite
the to-be-monitored function with a wrapper function which does the
logging and then calls the actual function.
This type of monitor is usually used for debugging purpose, and not for
security, because the monitoring can be circumvented.
\item In order to prevent circumvention,
the monitor can be executed in the kernel (assuming the kernel is authentic).
The \code{strace} utility uses the kernel \code{ptrace(2)} interface
to monitor system calls.
Most of the related work that we are going to introduce are kernel based.
\item In the same vein, to securely monitor kernel events, the monitor
should execute in a lower level than the kernel, i.e. the hypervisor.
Examples are the virtual machine monitors.
The recent Intel and AMD processors support hardware virtualization
features which can virtualize and monitor unmodified kernel with almost
no performance penalty.
\item The {\em instrumentation} technique is used to get instruction
level monitoring.
There are a number of instrumentation methods.
The earliest one executes binary program in a similar way
like interpreted language which parses and evaluates the instructions.
Later ones, such as Pin and Valgrind, alters the instruction stream on-the-fly
and let the CPU execute the altered instruction stream.
Section~\ref{sec:instrumentation} will discuss this in detail.
\end{itemize}

Another classification is whether the monitoring system is able to
alter the execution of the monitored software.
Logging systems such as \code{syslog} only record the events,
but do not alter the execution (except perhaps performance overhead).
%Other logging systems include Solaris BSM and DTrace.
\code{ptrace(2)}, on the other hand, can be used to filter system calls
or change system call's arguments.

\begin{table}
\centering
\begin{tabular}{|l|l|l|l|l|l|l|}
\hline
system & enforce & transp. & level & alter & OS & section\\ \hline \hline
\code{printf} & disc. & opaque & user & log & all & \ref{sec:printf} \\ \hline
\code{syslog} & disc. & opaque & user & log & POSIX & \ref{sec:syslog} \\ \hline
\code{ptrace(2)} & mand. & transp. & kernel & alter & POSIX & \ref{sec:ptrace} \\ \hline
\code{/proc} & mand. & transp. & kernel & alter & Solaris & \ref{sec:ptrace}  \\ \hline
Linux audit & disc. & opaque & kernel & log & Linux & \ref{sec:laudit} \\ \hline
FileMon & \multirow{3}{*}{mand.} & \multirow{3}{*}{transp.} & \multirow{3}{*}{kernel} & \multirow{3}{*}{log} & \multirow{3}{*}{Windows} & \multirow{3}{*}{\ref{sec:sysinternals}} \\
RegMon & & & & & & \\
ProcMon & & & & & & \\ \hline
DTrace & both & opaque & kernel & log & Solaris & \ref{sec:dtrace} \\ \hline
DProbes & \multirow{2}{*}{both} & \multirow{2}{*}{opaque} & \multirow{2}{*}{kernel} & \multirow{2}{*}{log} & \multirow{2}{*}{Solaris} & \multirow{2}{*}{\ref{sec:systemtap}} \\
SystemTap & & & & & & \\ \hline
Pin & \multirow{2}{*}{mand.} & \multirow{2}{*}{transp.} & \multirow{2}{*}{instru.} & \multirow{2}{*}{alter} & \multirow{2}{*}{Lin, Win.} & \multirow{2}{*}{\ref{sec:instrumentation}} \\
DynamoRIO & & & & & & \\ \hline
valgrind & mand. & transp. & instru. & alter & Linux & \ref{sec:instrumentation} \\ \hline
\end{tabular}
\caption{Classification of Monitoring Systems}
\label{tab:mon-tax}
\end{table}

Before discussing each related work in detail, Table~\ref{tab:mon-tax}
lists the classification of them.

\subsection{\code{printf}, Casual Debugging}
\label{sec:printf}

Directly printing debugging message to console
is probably the mostly widely used debugging technique
for simple programs, because of its portability and simplicity.
In C, \code{printf} is most commonly used for this purpose.
Other languages and environments have similar representatives
such as \code{System.out.println} in Java and
\code{printk} in Linux kernel.
However, this technique is not used in more sophisticated software.
Despite the reason that the debugging message can mess up
with the actual output, \code{printf} lacks the
separation between the monitor and the monitored program.
This means that monitoring output can be modified or removed by the monitored
program, thus a bug in the program can mess up the monitoring output.


\subsection{Traditional Syslog}
\label{sec:syslog}

Seeing the problem of \code{printf}, people developed the syslog
framework, which is the most widely supported logging framework for UNIX-like systems.
\code{syslog} separates log generation program and log recording program.
It works by letting the log generation program call the \code{syslog(2)}
system call and a dedicated daemon \code{syslogd} receive and record the log.
Kernel messages generated by \code{printk} is send to \code{syslogd} through
the middle man \code{klogd} in a similar way.

Although syslog protect the log from log generator,
we consider it as discretionary monitoring because it
requires the monitored program (log generator) to actively call
\code{syslog(2)} in order to generate a log message.
More specifically, if the monitored program is compromised,
it cannot modify already logged messages, but can suppress or spoof
new log messages.

\subsection{\code{ptrace} and \code{/proc}}
\label{sec:ptrace}

System call, the interface between user and kernel space,
is often monitored for various purpose.
The UNIX \code{ptrace} and Solaris \code{/proc} are commonly
used for system call monitoring because of their portability.
To use \code{ptrace}, the monitor calls the \code{ptrace(2)} system call
and specifies the process ID of another process to be monitored and waits.
When the monitored process makes a system call, the monitoring
process is waked up.
The monitoring process can then check or modify
system call parameters or return values.
The Solaris \code{/proc} works in a similar way except that
instead of calling \code{ptrace(2)}, it performs IO controls on the
\path{/proc/[pid]/ctl} file.
A subset system calls can be specified in \code{/proc},
while \code{ptrace} must monitor all system calls.

\code{ptrace} and \code{/proc} are mandatory monitoring
systems because the monitored program cannot evade the monitoring as long
as it make the system call.
The are transparent monitoring systems because in general,
the monitored program is not aware of the monitoring.
Because of this,
systems like Janus \cite{wagner1999janus} or
Alcatraz \cite{liang2009alcatraz} use them to do system call monitoring.
However, this usage is problematic because it is not
meant to be a secure monitoring mechanism, e.g.
\code{ptrace} was meant to support debuggers.
In the Solaris manual pages, \code{ptrace} is described as being
``unique and arcane''.
These kinds of problems and common
pitfalls with user-level system call interposition
are discussed in \cite{garfinkel2003traps}, such as:
(i) race conditions between time of check and time of use (TOCTOU), 
i.e. a buffer can be modified by another thread;
(ii) non-inheritance of tracing, i.e. special \code{strace} hacks in Linux;
and (iii) not transparent with respect to setuid/setgid executables 
and signals,
i.e. \code{ptrace} and \code{/proc} disable tracing 
on setuid/setgid executables.  
One could write a simple C program as in Figure~\ref{fig:ptrace-bug}
to escape from ptraceing.
After several iterations, some
child processes will not be traced by the tracing program.
As well as \code{ptrace(2)}, \code{/proc}
brings side-effect to the traced process.
When a traced process calls \code{setuid(2)},
the call will fail because the tracing process would have insufficient
privileges to the setuided process.
Because of their subtleties and intrinsic difficulties,
\code{ptrace} and \code{/proc} are not suitable for general purpose user-level
monitoring although they may be useful in specific situations.

\begin{figure}[htb]
\begin{verbatim}
#include <sys/types.h>
#include <sys/times.h>
#include <unistd.h>
main()
{
  struct tms timing;
  int i;
  getpid();
  if (fork() == 0) {
    getppid();
    if (fork() == 0) {
      times(&timing);
      _exit(1);
    }
    for (i=0; i < 10; getpid()) ;
  }
}
\end{verbatim}
\caption{Simple C program to escape \code{ptrace}}
\label{fig:ptrace-bug}
\end{figure}

The other serious drawback of \code{ptrace} or \code{/proc} is that
the overhead is considerable, incurring at least two
context switches per traced system call.
Our micro benchmarks in Section~\ref{sec:lbox-bench} show that
this can lead to an order of magnitude
slowdown on system call intensive programs.

\subsection{Linux Auditing System}
\label{sec:laudit}

The Linux auditing system (also known as lightweight auditing framework)
is used to monitor kernel events such as system calls and file system
operations.
The system consists of the kernel space event record producer and
the user space event record consumer (i.e. the audit daemon \code{auditd}).
At compile time,
kernel developers insert audit code into the kernel.
At run time, system administrators control which event and what information
to record using the \code{auditctl} tool.
All event records are transmitted through netlink sockets to \code{auditd}.
The event records are stored in a custom database which can be queried
using the \code{ausearch} and \code{aureport} tools.
The auditing system is incorporated into Linux kernel since version 2.6.4,
and is available in almost all Linux distributions.

The auditing system is a discretionary monitoring system to the kernel because
monitoring code is manually inserted by the kernel developer and can
be circumvented by kernel code.
However, when used for monitoring system calls of a user program,
it can be considered as mandatory if the kernel is assumed to be authentic.
The system only performs logging and does not alter the execution,
thus buffering of event records can be used to reduce context switches
and improve performance.

\subsection{Windows Sysinternals Tools}
\label{sec:sysinternals}

FileMon \cite{filemon} and RegMon \cite{regmon}
are file and registry monitoring tools for Windows, respectively.
They monitor operations taking place on
the registry or specified file system.
A graphical interface is used to filter and display monitored events in real time.
A later tool named Process Monitor combines features from both tools
and adds thread/process related event monitoring and event filtering.
The tools are closed source, so we study them by monitoring them
using our monitoring tool WinResMon (in Sec.~\ref{sec:winresmon}).
We found that they work by intercepting system calls and making
use of the kernel file system filter API.
Up on execution, a kernel driver is created in a temporary directory.
The driver is then loaded into the kernel and start to intercept
the kernel operations.
A named pipe is used to transmit event records.

The monitoring tools are standalone GUI programs, which do not
provide API to be used by other software.
We have observed that when events are fast generated, all tools
can drop events.
The details are covered in in Section~\ref{sec:resmon-overhead}.

\subsection{Solaris DTrace}
\label{sec:dtrace}

DTrace\cite{cantrill2004dynamic} is a dynamic tracing framework
created on Solaris 10, for troubleshooting kernel and application problems
on production systems.
Software developers insert probes into the code of the software
(kernel or user space program) at compile time.
System administrators or users monitor the execution by writing
a script in the {\em D language} and associating them with the probes,
so that when the software executes over the probes, the script is executed.
There are currently 30,000 \TODO{recheck} probes\footnote{
Since both Solaris kernel and DTrace are in active development,
our information is based on its current status in June 2011.}
in almost all aspects of the kernel.
The D script runs in the kernel and thus reduces the context switch.
For example, to count the number of \code{write(2)} system call of a process,
an integer variable is declared and a script which increments it
is associated with the \code{syscall::write:entry} probe.
Only one context switch is needed to output the final count.
To do this using \code{ptrace(2)}, a pair of context switch is
needed for each \code{write(2)} system call.

Having monitoring code dynamically (That is where the D comes from)
generated at runtime and executed in kernel is the key feature of
DTrace.
This poses a security threat as well however.
To prevent D script from running into infinite loops,
loops (or backward branch in general) and user defined functions
are not supported.

\TODO{Should I this mention here:
However, our micro-benchmarks in Section~\ref{sec:lbox-bench}
indicate that while it is significantly
faster than \code{/proc}, we have observed a factor of 2x slowdown.}

\subsection{SystemTap}
\label{sec:systemtap}
% http://sourceware.org/systemtap/wiki/SystemtapDtraceComparison

SystemTap is developed to bring the idea of dynamic instrumentation
in DTrace into Linux and possibly other UNIX variants.
Before we describe SystemTap, let us first look at its predecessor
DProbes and the underlying KProbes, which was introduced earlier than
DTrace.

KProbes is a Linux kernel framework which allows dynamic instrument
of kernel code.
The KProbes API consists of two components, the probe registration
functions and the probe activation functions.
The probe registration functions mark points in the code that can be
instrumented.
What it actually does is inserting a few \code{nop} instructions.
The probe activation functions associate registered code points
with probe handlers, which are called when the code points are executed.
What it actually does is rewriting the \code{nop} instructions with
a jump instruction targeting to the handler.
There are some optimization techniques, such as return address rewriting,
but the basic idea is the same.

KProbes is not convenient to use because its API are solely in kernel,
thus only kernel developers can use it.
DProbes is developed to allow user space program to make use of the
probe activation functions.
The way DProbes works is similar to DTrace, where a compiler
is used to compile a script which defines the probe handler,
and the compiler feeds the compiled script to the kernel
to execute.
What is different is that instead of compiling into intermediate
byte code as in DTrace, DProbes compiles directly into native machine code
which is feed to KProbes' activation functions.
The DProbes language is rather simple comparing to DTrace.
It is written in an assembly-like language,
based on the Reverse Polish Notation.
Logic, arithmetic and control flow operations are supported.
To prevent infinite loops, the number of branches is capped.

SystemTap also uses KProbes as the underlying kernel mechanism.
It uses a more advanced C-like language, where functions are
supported and a collection of library functions are provided.

\subsection{Instrumentation tool: Pin, valgrind and DynamoRIO}
\label{sec:instrumentation}
