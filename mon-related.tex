\subsection{Related Work}

Monitoring systems can be classified in a number of ways.
From the enforcement point of view, we have discretionary and
mandatory monitoring.
Discretionary monitoring requires that the monitored software
actively report to its monitor.
The traditional UNIX \code{syslog} is an example of discretionary monitoring.
A log entry is generated when the monitored software calls the \code{syslog()}
system call.
In contrast, mandatory monitoring systems enforce that logs are always
generated when certain actions are performed by the monitored software.
The \code{ptrace(2)} interface and Solaris Basic Security Module (BSM) Auditing
are examples of mandatory monitoring.
Mandatory monitoring is more suited for security purpose because of the enforcement.
Discretionary monitoring may give more friendly output since the monitored
software knows which pieces are more important.

A related classification is transparent/opaque monitoring.
In transparent monitoring, the monitored software does not need to be adapted
and sometimes does not aware of being monitored;
whereas in opaque monitoring, the monitored software need to be either
rewritten and recompiled or transformed manually.
The two types of classification are related because transparent monitoring
is usually mandatory as well, and opaque monitoring is usually discretionary.
\TODO{add exception}

From OS point of view, the monitoring system can be in user space, kernel space
or both.

\subsubsection{Traditional \code{syslog}}

Syslog is the main logging tool on a UNIX systems. It centrally manages logging
for your main system services. Syslog is discretionary in such a sense that it
requires the program to actively call the \code{syslog(2)} system call in order to
generate a log message. Syslog is implemented in two halves. One half is a set
of standard C-library routines. Programs utilize these routines to send
messages to the log daemon. The other is a daemon process \code{syslogd}. It is
configured with the \path{/etc/syslog.conf} file. It writes log messages to various
locations.

When writing kernel code, one can send log messages to \code{syslog} by calling
\code{printk}\footnote{\code{printk} is Linux kernel specific} kernel.  This is
realized through \code{klogd}, the kernel log daemon.  \code{klogd} is a system
daemon which intercepts and logs Linux kernel messages. It is usually
configured to forward kernel messages to \code{syslogd}. Traditional logging
mechanism such as \code{syslog} is discretionary in nature, thus cannot be used
for security purpose. In the security monitoring context, we should not trust
the monitored program, thus the log message supplied by it should also not be
trusted. For example, the application does not log or logs something to spoof
the monitor.

\subsubsection{\code{ptrace}}

The \code{ptrace(2)} system call provides a means by which a parent process may
observe and control the execution of another process, and examine and change
its core image and registers. It is primarily used to implement breakpoint
debugging and system call tracing. Most UNIX systems, such as Solaris, FreeBSD,
and Linux implement this system call.

While being traced, the child will stop
each time a signal is delivered, even if the signal is being ignored. The
parent will be notified at its next \code{wait(2)} and may inspect and modify the
child process while it is stopped. The parent then causes the child to
continue, optionally ignoring the delivered signal (or even delivering a
different signal instead).

There are many disadvantages on \code{ptrace} as an system
auditing tool. Firstly, it is very inefficient. Since the child process will
stop at each system call, there will be a high process switch cost on each
system call. Further more, the process still has to stop at every system call
even if we only want to audit one specific system call. Secondly, if the
process is already been traced by some process. It is not possible for another
process to trace it. This is not desirable as an auditing tool, because
sometimes we need different processes to audit on different things. Lastly,
\code{ptrace} will bring side-effect to the process been traced. For example, if a
process is attached to with PTRACE ATTACH, its original parent can no longer
receive notification via wait when it stops, and there is no way for the new
parent to effectively simulate this notification. This is also not desirable
because we want the process behave the same as while not been traced.

\begin{figure}[htb]
\begin{verbatim}
#include <sys/types.h>
#include <sys/times.h>
#include <unistd.h>
main()
{
  struct tms timing;
  int i;
  getpid();
  if (fork() == 0) {
    getppid();
    if (fork() == 0) {
      times(&timing);
      _exit(1);
    }
    for (i=0; i < 10; getpid()) ;
  }
}
\end{verbatim}
\caption{Simple C program to escape \code{ptrace}}
\label{fig:ptrace-bug}
\end{figure}

\code{ptrace}
suffers from many race condition problems\cite{garfinkel03tap}. One could write a simple C
program as in Figure~\ref{fig:ptrace-bug} to escape from ptraceing. After several iterations, some
child processes will not be traced by the tracing program.

Many Linux kernel
exploits make use of \code{ptrace} bugs. The famous ones are "Linux Non-Readable File
Ptrace Vulnerability" in 2000, "Linux Ptrace/Setuid Exec Vulnerability" and
"Linux ptrace/execve Race Condition Vulnerability" in 2001.

\subsubsection{Solaris /proc}

Solaris \path{/proc}\cite{faulkner1991tpf}\footnote{Different UNIX systems implement
different \path{/proc} file systems.}
is a file system that provides access to the state of each process
in the system. \path{/proc} can be used for system call interposition.

\begin{figure}[htb]
\begin{verbatim}
-rw------- 1 wuyongzh grad 4333568 May 10 16:49 as
-r-------- 1 wuyongzh grad     152 May 10 16:49 auxv
dr-x------ 2 wuyongzh grad      32 May 10 16:49 contracts
-r-------- 1 wuyongzh grad      32 May 10 16:49 cred
--w------- 1 wuyongzh grad       0 May 10 16:49 ctl
lr-x------ 1 wuyongzh grad       0 May 10 16:49 cwd ->
dr-x------ 2 wuyongzh grad    8208 May 10 16:49 fd
-r--r--r-- 1 wuyongzh grad     120 May 10 16:49 lpsinfo
-r-------- 1 wuyongzh grad     912 May 10 16:49 lstatus
-r--r--r-- 1 wuyongzh grad     536 May 10 16:49 lusage
dr-xr-xr-x 3 wuyongzh grad      64 May 10 16:49 lwp
-r-------- 1 wuyongzh grad    3168 May 10 16:49 map
dr-x------ 2 wuyongzh grad     800 May 10 16:49 object
-r-------- 1 wuyongzh grad    3848 May 10 16:49 pagedata
dr-x------ 2 wuyongzh grad    9008 May 10 16:49 path
-r-------- 1 wuyongzh grad      72 May 10 16:49 priv
-r--r--r-- 1 wuyongzh grad     336 May 10 16:49 psinfo
-r-------- 1 wuyongzh grad    3168 May 10 16:49 rmap
lr-x------ 1 wuyongzh grad       0 May 10 16:49 root ->
-r-------- 1 wuyongzh grad    1536 May 10 16:49 sigact
-r-------- 1 wuyongzh grad    1232 May 10 16:49 status
-r--r--r-- 1 wuyongzh grad     256 May 10 16:49 usage
-r-------- 1 wuyongzh grad       0 May 10 16:49 watch
-r-------- 1 wuyongzh grad    5016 May 10 16:49 xmap
\end{verbatim}
\caption{List of the \code{/proc/pid} directory in Solaris 5.10}
\label{fig:proc}
\end{figure}

The name of each entry in the \path{/proc}
directory is a decimal number corresponding to a process-ID. These entries are
themselves subdirectories. Access to process state is provided by additional
files contained within each subdirectory. The owner of each \path{/proc} file and
subdirectory is determined by the user-ID of the process.

There are many files under each \path{/proc/pid} directory as shown in
Figure~\ref{fig:proc}. Most files describe process state and can only be opened
for reading. Address space files contain the image of the running process and
can be opened for both reading and writing. An open for writing allows process
control; a read-only open allows inspection but not control.

In general, more than one process can open the same \path{/proc} file at the
same time. Exclusive open is an advisory mechanism provided to allow
controlling processes to avoid collisions with each other. A process can obtain
exclusive control of a target process, with respect to other cooperating
processes, if it successfully opens any \path{/proc} file in the target process
for writing while specifying \code{O\_EXCL} in the \code{open(2)}. Such an open
will fail if the target process is already open for writing.

Data may be transferred from or to any locations in the address space of the
traced process by applying \code{lseek(2)} to position the as file at the
virtual address of interest followed by \code{read(2)} or \code{write(2)} (or
by using \code{pread(2)} or \code{pwrite(2)} for the combined operation). The
address-map file \path{/proc/pid/map} can be read to determine the accessible
areas (mappings) of the address space.

A tracing process writes specially-formatted commands to the ctl file to
specify events of interest and actions to be applied to the tracing process.
These commands appear similar to the arguments given to the \code{ptrace(2)} call: an
identifier indicating the requested control operation and any argument required
for that operation. We will describe some commands which are related to system
auditing.

\begin{enumerate}
\item \code{PCSTOP}: When applied to the process control file, PCSTOP directs
the process to stop and waits for it to stop.
\item \code{PCRUN}: Make a process runnable
again after a stop. 
\item \code{PCSTRACE}: Define a set of signals to be traced in the
process. The receipt of one of these signals by a process causes the process to
stop.
\item \code{PCKILL}: If applied to the process control file, a signal is sent to
the process with semantics identical to those of kill(2).
\item \code{PCSENTRY}/\code{PCSEXIT}:
These control operations instruct the process to stop on entry to or exit from
specified system calls. When entry to a system call is being traced, a process
stops after having begun the call to the system but before the system call
arguments have been fetched from the process. When exit from a system call is
being traced, a process stops on completion of the system call just prior to
checking for signals and returning to user level. At this point, all return
values have been stored into the process' registers.
\item \code{PCWATCH}: Set or clear a
watched area in the controlled process from a prwatch structure operand. A
watch-point is triggered when a process in the traced process makes a memory
reference that covers at least one byte of a watched area and the memory
reference is as specified in pr wflags. When a process triggers a watch-point,
it incurs a watch-point trap.
\end{enumerate}

Comparing to \code{ptrace(2)}, \path{/proc} gives a finer
grained control over monitored system calls. We can specify only the interested
system calls to intercept. This can give a great performance improvement if we
want to monitor a small set of system calls. Another advantage over \code{ptrace(2)}
is that a process can be traced by more than one tracing process using /proc.
However, \path{/proc} still suffers from performance penalty caused by process switch
on each traced system calls. It requires the tracing process' immediate
response on each traced system calls. As a result, it can be used for process
confining but not good for auditing.

As well as \code{ptrace(2)}, \path{/proc} will still
bring side-effect to the traced process. When a traced process calls \code{setuid(2)},
the call will fail because the tracing process would have insufficient
privileges to the setuided process.

\subsubsection{Light-weight Auditing Framework}

The lightweight auditing framework is intended to be a way for the kernel to
get various types of audit information out to user space without slowing things
down, especially when auditing is not being used. The framework is meant to
serve as a complement to SELinux; it is already being shipped as a part of the
Fedora Core 2 test 2 kernel. There are two kernel-side components to the audit
code. The first is a generic mechanism for creating audit records and
communicating with user space. All of that communication is performed via
netlink sockets; there are no new system calls added as part of the audit
framework. Essentially, a user-space process creates a NETLINK AUDIT socket,
writes audit request structures it, and reads back audit reply structures in
return. The generic part of the audit mechanism can control whether auditing is
enabled at all, perform rate limiting of messages, and handle a few other
tasks. On the kernel side, it provides a printk()-like mechanism for sending
messages to user space. This code also implements a user-specified policy on
what happens if memory is not available for auditing; truly paranoid
administrators can request that the kernel panic in such situations.

\subsubsection{Linux Trace Toolkit (LTT) and LTTng}

\subsubsection{Solaris DTrace}

\subsubsection{KProbes, DProbes and SystemTap}
