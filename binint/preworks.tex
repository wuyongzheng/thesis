\subsection{Related Work}
\label{sec:binint-prevworks}


\subsubsection{Isolation Models}

Isolation can be used to prevent malware from changing
other parts of the system.
A straight forward form of isolation is to have 
multiple independent systems or virtual machines (VM).
This is useful for protecting one application against another though
it might not protect against the PDF and WinHelp attacks discussed in
Sec.~\ref{sec:binint-usageattack} which are against the applications themselves.
There is also the question of whether it is effectively feasible to 
run each software within its own VM.
On the one hand,
this by definition also prevents sharing which is incompatible
with the use of implicit sharing by software in Windows and
may be incompatible with system services and utilities.
It also means a loss of usability since data is not sharable
between applications.
The isolation also leads to high
maintenance cost of updating all VMs against
vulnerabilities as there are multiple copies of
the same software or library.
On the other hand, using isolation separately on each software can also
have significant system overheads since large numbers of virtual machines
need to be run.

One-way isolation is a form of isolation which
lets untrusted programs run in a sandbox where they can read from the
trusted base system but modifications are confined to the sandbox.
In one-way isolation, processes can be executed in isolation domains,
other processes are executed normally in the base system.
The semantics of file access is modified such that an isolated process
can read files in the base system but when it tries to modify them,
a private copy is duplicated from the base system and subsequent modification
is applied on the private copy.
Special care must be taken to handle file deletion so that files in the
base system are not accessed.

Kato et al. proposed an one-way isolation system named
SoftwarePot \cite{kato2003softwarepot},
whose main purpose is to allow software to securely circulate among
different hosts.
Secure circulation means that software in the sandbox should not interfere
with the base system.
A security policy specifies files visible to the sandbox,
path mapping from base system to sandbox, system calls allowed and
network addresses allowed to interact.
Liang et al. proposed another one-way isolation system named
Alcatraz \cite{liang2009alcatraz}.
The main difference between SoftwarePot and Alcatraz is that the latter
enables modification from the sandbox to be {\em committed} to the base
system.

The problems of one-way isolation are similar to that of
virtual machines, namely, that isolation is the opposite of sharing.
Other than that, one-way isolation protects the integrity of non-isolated
system from attackers in the isolated system,
but it does not protect the integrity of the isolated system.
There are many forms of interactions between the sandbox and the base system,
such as file system access, networking and IPC
(including local sockets, pipes, process synchronization mechanisms,
signals, window messages, etc.).
The first two forms are easier to sandbox because their semantics are known.
However, IPC is application dependent and can be complicated and
undocumented.
Even if only considering documented ones,
it would take too much effort to implement filters for different IPC
protocols (e.g. shared memory, pipe, CORBA, D-Bus, SunRPC etc.).
SoftwarePot does not discuss IPC and Alcatraz simply disallows some forms of IPC.
However, IPC is important and cannot be ignored.
For example, the X window system, is implemented as local sockets;
the Windows GUI requires (undocumented) IPC to the {\tt smss.exe}
and {\tt csrss.exe} processes.
Some services are implemented as IPC such as
service management and domain name resolution in Windows,
PulseAudio (the sound system in Linux),
and syslog (the UNIX logging service).
Simply allowing and denying them will lead to either insecure sandboxing or
unusable software.

\subsubsection{Signed Binaries}
\label{sec:binint-signing}

There are systems \cite{halim2008lightweight,apvrille2004digsig,williams2002anti,doorn01signedexecutables,nanda2006fcd}
that only allow signed binaries to be loaded or executed.
In these systems, signature of binaries are either embedded
in the files themselves or signed by a third party, such
as the administrator and stored in a secure database.
Executing or loading are only allowed if the signature can be verified.

There are three basic problems with signed binaries.
Firstly, the security relies on the trust of the signing keys.
If key is leaked, the trust is broken.
For example, the Stuxnet worm \cite{matrosov2010stuxnet} uses
signed drivers with real certificates which are trusted by Windows.
Secondly, revocation checking is expensive as it cannot be done
locally and furthermore may not be timely.
Lastly, we cannot assume that all software vendors sign their binaries.
A more important difference is that most signed binary schemes are not
intended to deal with the software lifecycle issues which we are concerned with.
Another difference is that signed binaries
only ensures that the binaries are not modified but
may not prevent deletion.

\paragraph{Self-signed executables}

Self-signed executables \cite{wurster2007self} is a different approach
to signed binaries which allows for software updates.
In their proposal, all binaries are embedded with their signatures signed
by the software vendor.
Unlike the previous signed binary systems, the assumption is that any
binary which has been signed can be trusted.
Using the observation that binaries are usually updated by a newer
version from the same vendor as the original,
binary updating is allowed if the signature can be verified
with the same public key.

The self-signed executables system binds paths with keys.
Once bound, the key cannot be changed, even when the file is removed.
An attacker can also sign malware as long as it is the first
software installed with that path.
Alternatively, this can constitute a denial of service attack since
an attacker binding the path first would disallow future correct
software to be installed at that path.
As deleted files leave behind a stub, the number of stubs will
monotonically increase over time. This can lead to yet another
denial of service attack to exhaust storage.

In order to compare the signatures between the old and new copy,
semantics of file writing has to be changed such that
writing is applied on a shadow copy and later updated atomically.
This can break software that assumes normal POSIX semantics.
