\subsection{Windows Binary Security Issues}
\label{sec:problems}

The Windows NT operating system
is difficult to secure
because of its closed source and complex nature.
To understand the issues, we first discuss how
binaries are used in Windows followed
by some representative attacks.

\subsubsection{Binary Usage in Windows}

A {\em binary} in Windows is a file containing executable code.
Binaries in Windows occur in many contexts,
the following is a non-exhaustive list:
applications/executables ({\tt exe}),
command files ({\tt com}),
Dynamic Linked Libraries ({\tt dll}),
ActiveX controls ({\tt ocx}),
device drivers ({\tt drv},
screensavers ({\tt scr}),
control panel applets ({\tt cpl}),
Input Method Editors ({\tt ime}),
and codecs ({\tt acm}, {\tt ax}).
Obviously a binary needs to be in the correct format.
In Windows, the only way to identify a binary
is its format, the filename extension is only a convention.

A single process in Windows typically uses many binaries which
can originate from many sources.
For example, a software installation may introduce many binaries including
shared libraries which can be loaded by many/all processes.
To illustrate, even a small program like {\tt notepad}
loads 13 binaries from 3 different directories.
(It happens here that because the AVG antivirus is installed, the
binary {\tt avgrsstx.dll} is also loaded).
In contrast, {\tt Internet Explorer} on a {\em YouTube} page may load
86 DLLs from {\tt System32}, 14 other DLLs, 3 drivers and some other
binaries.
A more detailed look at the sharing and dependencies of Windows binaries
can be found in \cite{wu2010comprehending}.

There are numerous mechanisms in Windows which can cause a binary to be used,
i.e. load the binary.
Subsequently, code in the binary may be executed.
Some of the common mechanisms used both by legitimate software
as well as malware are:
DLL injection, DLL search order hijacking, Autorun/Autoplay, Autostart,
shell extensions, Windows Services, {\tt rundll\linebreak[0]32.exe},
and Browser Helper Objects.

\subsubsection{Normal Usage versus Malicious Attacks}
\label{sec:usageattack}

A security model must be able to distinguish
malicious attacks from normal software usage.
We firstly look at typical software usage scenarios in the
software life cycle, namely running, installing, updating and developing 
a software.
We then look at some attacks which exploit binaries whose
behavior is similar to the normal usage scenarios.

The most common scenario is that of {\em running} a software.
Normal usage typically involves loading binaries,
reading/writing data files and possibly the creation of child processes.
It is, however, not common for binaries to be created or modified.
We highlight that binaries can come from
various (3rd party) sources.
For example, Windows Explorer (Windows GUI shell) loads 3rd party DLLs
to provide shell extensions.
The shell extensions are used to preview
images or video, to add a control panel component, to customize
file icons and context menus, etc.
Another example is anti-virus software which forces
its DLLs to be loaded by {\em all} processes, e.g. {\tt avgrsstx.dll}
from the AVG anti-virus software.

{\em Installing} software involves creating binaries
and data files.
Installers downloaded from the Internet are typically packed as
self-extracting archives, which
unpack themselves into a temporary directory and launch from there.
This unpack-and-run behaviour is also observed in the normal
running of some software, e.g.
many {\tt Sysinternals} tools
unpack a kernel driver into a temporary directory and then load it.

There are a number of software {\em update} mechanisms.
Microsoft Windows Update employs a dedicated service (daemon process) to check,
download and apply updates.
In this case, the updater is completely separate from 
the software being updated.
Other software, such as {\tt Mozilla Firefox} and {\tt Sun JDK},
employ a self-update mechanism.
We use Firefox to illustrate the mechanism.
Firefox checks online if there is a new version available and
downloads the update as a compressed archive file.
After downloading, Firefox invokes the updater
as a helper process and terminates itself.
The updater process reads the downloaded archive and 
updates {\tt firefox.exe}
and other files if necessary.
When updating is finished, the updater re-launches Firefox ({\tt firefox.exe})
and terminates itself.
The helper process technique is usually used
because in Windows, a binary cannot be written to when it is loaded by
another process.

Software {\em development} means that binaries have to be created.
For example, to debug a software using an IDE,
the IDE firstly builds the program and then runs it.
Thus the IDE creates or modifies binaries then executes them.

We now turn to some common attacks that involve binary loading or modification.
Furthermore, these attacks use binary vulnerabilities in trusted programs.
We highlight how the attacks behave in a similar fashion to the normal usage
scenarios described earlier and the security mechanism needs
to distinguish between normal usage versus an attack.

\paragraph{Running Unintended Executables}

A common attack is to employ social engineering techniques to get
the user to run a malware executable.
However, the attack can be more subtle and exploit features of various
applications to hide the executable. 
For example, by default, Windows Explorer hides file extensions. 
Consider a malware {\tt postcard.jpg.exe} where
the icon of the malware also looks like a photo.
The user is then fooled into thinking it is a JPEG image and click
it for viewing in Windows Explorer.
The malware could execute its payload and then display
the photo.
Thus, the user would probably think this is just the normal display
of a JPEG image.
Notice that the behaviour of both a direct or file extension social
engineering attack is similar to the regular
software running scenario.

\paragraph{Application Data with Embedded Binaries}

Malware binaries can be embedded within documents.
A recent vulnerability in PDF readers \cite{pdfattack} illustrates this
attack where a legal PDF file containing an embedded executable.
When the PDF file is viewed, the PDF viewer (e.g. {\tt Acrobat}) also
automatically executes the embedded executable.
Notice that the behavior in the attack
resembles the unpack-and-run in the normal installation scenarios.


\paragraph{DLL Attacks}

DLL attacks typically exploit some feature of an application
to inadvertently load a malicious DLL.
The ``carpet bomb'' \cite{safari-carpet-bomb} attack involves bugs of two different
browsers.
Safari had a feature\footnote{
Apple did not consider it as a bug.
}
which downloaded files onto the user's desktop automatically.
Internet Explorer had a bug where certain DLLs are specified
by the file name rather than a full pathname.
By combining the behavior of Safari and Internet Explorer,
a malicious website can execute arbitrary code on a machine 
that visits the site.
It is surprising to see that this vulnerability where applications
do not use a sufficiently qualified path when loading DLLs
has recently re-surfaced
(in Aug 2010) as ``binary planting'' attacks \cite{binary-planting}.

In some cases, an application has features to execute binaries.
This can be exploited.
A vulnerability \cite{winhelp-attack} in handling Windows Help files in
Internet Explorer was recently discovered.
A malicious web page can cause Internet Explorer to load a malicious
WinHelp ({\tt .hlp}) file from an arbitrary path.
WinHelp also has a feature to allow loading of arbitrary DLLs.
The bug and the feature together enable a malicious website to execute
arbitrary code on the client.

The recent Stuxnet worm \cite{matrosov2010stuxnet} also uses a DLL attack.
Interestingly, Windows 7 does not even give a UAC warning with this attack.
Stuxnet exploits a vulnerability in Windows Explorer which displays an icon
for shortcut files by using a DLL specified in the shortcut.

All above attacks load existing malicious DLLs which could be accidentally
downloaded/copied or exist in a network drive.
Once the DLL is in place, the behaviour of the attacks is similar to the
regular software running scenario.
