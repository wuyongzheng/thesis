\subsection{Implementation}
\label{sec:binint-imp}

We implemented BinInt in Windows XP.
We have a kernel driver that intercepts binary loading,
file modification and some other operations.

In Sec~\ref{sec:binint-model}, we assume all file modifications are under
the control of the operating system kernel.
This assumption can be invalid in some cases.
When the system mounts a network shared file system, (e.g. through SMB)
an attacker can change the binaries outside the system.
We call these unmonitorable binaries.
An attacker can change local binaries when the system is offline,
or change binaries in removable media when it is offline.
To prevent these attacks, we use file signatures to detect modification.
We keep signatures of all b-valid binaries in the
{\em binary signature database},
which is similar to BinAuth.
Information in the database includes path, signature, software group,
modification history and other meta data.
We also generate logs of how binaries were used. Log maintenance is done
outside the kernel.
The database can be used to by tools to analyze binary dependencies,
trace the origin of a binary, etc.

For unmonitorable b-valid binaries, their signatures are updated
immediately after they are updated and verified every time when
their labels are accessed.
For binaries that just come online, we verify the signatures
once for each binary and cache the result.
We adopt a lazy way of updating their signatures in order
not to affect performance.

The kernel driver also maintains the signatures of binaries and
labels of processes and binaries.
When a binary $f$ is loaded by a process $p$ running in d-mode,
$f$'s needs to be checked whether it is b-valid.
We intercept the native call {\tt NtCreateSection},
which is a necessary step in binary loading.
Checking whether $f$ is b-valid is achieved by calculating 
the signature of the contents
of $f$ and match against the signature database, if $f$ is loaded for the first
time since boot or $f$ is in an unmonitorable media.
Otherwise, the software domain of $f$ can be looked up using its path.
Paths are normalized into
the internal kernel path to disambiguate 8.3 filename, long file name
and symbolic links.
For NTFS, we use object ID to disambiguate hard links.
If binary $f$ is b-invalid,
the loading will be denied, thus {\tt NtCreateSection} fails.
% As we implement a binary loading mechanism very close to \cite{binauth},
% the reader is directed there for details on
% the security of binary loading
% and managing the caching optimizations.

The rest of the implementation deals other modes
and maintaining integrity of binaries.
If the process is in i-mode or t-mode, the checking
will be bypassed and the loading will always be allowed.
When a process $p$ tries to open a b-valid binary
$f$ with write permission, (i.e. try to modify)
the open operation is only allowed if $p$ is in i-mode
and the software domain of $p$ and $f$ are the same.
As there is no distinguishing feature of a binary, other than its format,
we need to immediately test whether or not the file is a binary
by reading the file header which  makes i-mode more costly
than other modes.
We also modify the semantics of Windows slightly so that files opened
for writing in i-mode are in exclusive mode to simplify signature creation.
We do not expect this to be a major restriction as the installer is likely
to be creating files sequentially.
When $p$ closes the file handle of $f$, the file contents is now complete
and $f$'s signature will be re-computed and the signature database will 
be updated accordingly.
The signature re-computation is done immediately if $f$ is in 
un-monitorable media.
Otherwise, it is lazily performed at any time before system shutdown since
its only needed on next boot.
Note that, to remove or rename a file in Windows, a process has to open the file
with write permission first.

We do not prevent creation of new binaries,
but new binaries created by processes in d-mode or t-mode will not
be b-valid and thus cannot be loaded in d-mode.

Process creation is intercepted using
the kernel API {\tt PsSetCreateProcessNotifyRoutine} to inherit e-mode and
s-domain to child processes.
The systemcall-like {\tt modetrans} operation
is implemented as a IOCTL (I/O control).

A software installer may launch
several helper programs from the main one.
This is why child processes inherit their parent process' mode
and domain.
However, there are exceptions where installer processes
are not child (or descendant) processes of the first installer process.
MSI (Windows Installer) is a generic installation engine
for installing and updating software on Windows.
It is commonly used for Microsoft and non-Microsoft software.
MSI makes use of a service (daemon) process to perform installation.
The service process is always running and is not part of the
process hierarchy of the original installer.
We handle this by monitoring the communication channel, 
a named pipe \url{\\Pipe\\Net\\NtControlPipeX}, between the
installation process and the MSI service.
When an i-mode process triggers the service to start installation,
the service is switched to i-mode with the same domain as the triggering
process.
When the installation terminates, the service is switched back to d-mode.
The MSI service is used exclusively so that there is no interference between
concurrent requests.
