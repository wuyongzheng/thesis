\subsection{The BinInt Security Model}
\label{sec:binint-model}

In Section~\ref{sec:binint-prevworks}, we discussed a number of existing
security models. While those models can be used to provide security
for binaries, the main drawback is that they do not cover the usage spectrum
across the software lifecycle nor do they handle all the
typical attack scenarios mentioned.
We now present a new security model for binaries
which we call BinInt.
This model combines file integrity together with
aspects of signed binaries and isolation to give a binary
security mechanism which
addresses usability in the software lifecycle. We remark
that while we focus on Windows, this model has general applicability
to other operating systems.

We are cognizant that a security mechanism
should provide the desired form of protection in practice for
the intended system usage.
Full two-way isolation protects the host
against malware in the VM but may not
protect against other attacks.
More importantly, it does not protect the software running
in the VM from malware.
This is why our model combines a number of features together to provide
binary integrity.
We focus solely on integrity and security of binaries
as want to preserve compatibility with existing
software (in Windows) while allowing implicit sharing.
Our model should not be considered as a sole security mechanism and
is meant to be used with other security mechanisms,
e.g. ACLs or MAC, IDS, etc.

Our goals are three fold.
Firstly, to prevent attacks exploiting vulnerabilities which load
untrusted binaries.
Secondly, to ensure integrity of binaries.
This both prevents malware from being installed on the system and
prevents denial of service attacks which delete binaries.
Thirdly, we want to achieve a middle ground
between usability and security.
Any changes to system usage should not be onerous and existing software
should (mostly) be able to run, including closed source binaries.
While it is impossible to guarantee that all unmodified software can still
function normally together with a security mechanism, our goal is that
it should work in practice for typical software.
Thus, frequent tasks, such as running and updating benign software,
should be (mostly) transparent.
Other infrequent tasks, such as software installation and removal,
can be a little different but should not cause much inconvenience.
As the mechanism should be applied to all processes,
it should only have small overheads especially for frequent tasks.

We now describe our BinInt model.
All processes are in one of the following execution modes:
{\em d-mode} (default mode), {\em i-mode} (install mode) and
{\em t-mode} (temporary trusted mode).
Intuitively, d-mode corresponds to running already installed software,
while i-mode is for installing/updating software.
Special cases for running software
which needs to dynamically create and load binaries but is not a software
install are handled by t-mode,
e.g. building and running binaries in an IDE
or dynamic temporary binaries as in Sysinternals tools.

Each process and binary is associated with a {\em software domain}, which
relates the process or binary to a particular installed software.
The idea is that we may use the name of a particular software or
the software vendor as the software domain.
A special software domain $\bot$ denotes binaries
which do not have a valid software domain.
Binaries whose software domain is not $\bot$ are called {\em b-valid}.

Now we describe the BinInt model using the following rules.
Rules 1--2 enforce loading of binaries;
rules 3--6 enforce creation, modification, and deletion of binaries;
and rules 7--8 define the execution modes for processes.
% We use {\em EXE} to denote the executable binary of a process.

\begin{description}
\setlength{\itemsep}{0pt}
\setlength{\parskip}{0pt}
\setlength{\parsep}{0pt}
\item[R1]
A d-mode process cannot load b-invalid binaries,
i.e. can load only b-valid binaries.
Intuitively, this means that untrusted binaries cannot be loaded
by normal processes.
\item[R2]
An i-mode or t-mode process can load all binaries.
\medskip
\hrule width \textwidth
\medskip
\item[R3]
A d-mode or t-mode process cannot modify or delete b-valid binaries,
i.e. can modify or delete only b-invalid binaries.
Intuitively, this means that trusted binaries cannot be modified
by normal processes.
\item[R4]
An i-mode process can modify or delete a b-valid binary
if their software domains are the same.
It can also modify or delete b-invalid binaries.
Intuitively, this means that an installer or updater should update
only its {\em own} binaries.
% It should not mess up other software.
\item[R5]
A new binary created by d-mode or t-mode process is b-invalid.
\item[R6]
A new binary created by i-mode process is b-valid and the binary's software domain is
the same as the process.
This rule together with rule R5 means that only an installer
should create binaries.
Instead of completely preventing normal process from creating binaries,
we choose to label those binaries b-invalid. This is because it is
not possible to determine during file operations 
whether a file is a binary or not.
\medskip
\hrule width \textwidth
\medskip
\item[R7]
When the system is booted, the initial process(es) run in d-mode.
When a new process is created, the execution mode and software domain
inherits from its parent process.
There are two mechanisms to change the mode and domain.
The first mechanism is {\em execution mode policy},
which maps the path of an EXE to a corresponding execution mode and software domain.
When a new process is created, the path of the EXE is checked against the
execution mode policy to find the execution mode and software domain.
We require that the EXE should be b-valid.
If no such mapping is found in the execution mode policy, 
it inherits the execution mode and software domain of its parent process.
The execution mode policy is defined by a protected configuration file.
\item[R8]
The other mechanism changes the mode and domain of a running process by
performing a special operation similar to a system call.
In order to ensure that only the user with appropriate privileges
can change the execution mode and software domain, the operation is authenticated with
a password sent to the kernel.
% It requires authentication (a password is passed to the kernel) to
% perform the operation.
We have implemented a {\tt sudo}-like utility called {\tt modetrans}
using this operation.
{\tt modetrans} takes the password from the user and
executes a user-specified program in a user-specified execution mode and software domain.
% In addition, the specified program can optionally be required to be b-valid
% (see also the interaction with BSA in Section~\ref{sec:database}).
Although {\tt modetrans} is a command-line program,
a more user friendly GUI version can be implemented to
integrate with the Explorer shell, so that the user can right-click
an EXE and choose ``run in install mode''.
% \TODO{add b-valid to modetrans}

Both execution mode policy and {\tt modetrans} are used to run an EXE in a
specified execution mode and software domain.
The execution mode policy is preferred if the EXE should always run
in that execution mode and software domain.
For example, since the Windows auto-updater {\tt wuauclt.exe} should
always run in i-mode and with ``Microsoft'' software domain, the execution mode policy
should have an entry which maps \verb|C:\windows\system32\wuauclt.exe|
to i-mode and ``Microsoft'' software domain.
{\tt Modetrans} has an option to require that the EXE is b-valid
before execution.
This is useful when we want to guarantee that the installer
is trusted before it is allowed to create new trusted binaries.
% If the examination is turned on, b-invalid EXE cannot be executed even if it
% is specified in the execution mode policy or {\tt modetrans} argument.
% \TODO{need to discuss the purpose of the option?}
\end{description}

% We now describe our BinInt model.
% All processes are in one of the following execution modes:
% {\em d-mode} (default mode), {\em i-mode} (install mode) and
% {\em t-mode} (temporary trusted mode).
% Intuitively, d-mode corresponds to running already installed software,
% while i-mode is for installing/updating software.
% Special cases for running software
% which needs to dynamically create and load binaries but is not a software
% install are handled by t-mode,
% e.g. building and running binaries in an IDE
% or dynamic temporary binaries as in Sysinternals tools.
% 
% Each process and binary is labeled with a {\em software domain}, which
% relates the process or binary to a particular installed software.
% The idea is that we may use the name of a particular software or
% the software vendor as the software domain.
% A special software domain $\bot$ denotes binaries
% which do not have a valid software domain.
% Binaries whose software domain is not $\bot$ are called {\em b-valid}.
% 
% We now describe the rules.
% Creating a binary in i-mode sets its label to the software domain
% of the process and to $\bot$ otherwise in d-mode or t-mode.
% A binary can only be written to or deleted in i-mode provided
% it has the same software domain as the process or it is $\bot$.
% In d-mode or t-mode, only binaries whose label is $\bot$ can be written to.
% There are no restrictions on the read operation as it has no effect on binary
% loading and integrity.
% 
% A binary can be loaded/executed in d-mode only if it is b-valid.
% As t-mode and i-mode need to use new binaries, there are no restrictions
% on loading/execution.
% Rather the restriction is on the privilege needed to change to t-mode or i-mode.
% The mode and software domain is preserved when creating a new process.
% In our model, except for $\bot$,
% the software domain is only used in i-mode.
% 
% In order to change the mode of a process, we define a
% special operation, called {\em modetrans}.
% Modetrans is meant to be a privileged operation, and as such,
% requires user authentication and appropriate privileges, i.e.
% similar to {\tt sudo} in Unix or UAC in Windows.
% As i-mode, is meant for new software, the user (or the execution mode policy)
% needs to also specify what is the (new) software domain.
% Note that our model assumes all files are typed as binary or not,
% which means that binary files can only be written as data files in
% i-mode.
% This is not a significant restriction since binaries have a special format
% and are usually created/written to in a special context where there is
% an easy correspondence to one of our modes.
% 
% It is easy to map BinInt to the scenarios in Sec~\ref{sec:binint-usageattack}.
% Installed software normally runs in d-mode;
% software install/up\-date/uninstall employ i-mode;
% and special cases like the IDE run in t-mode.
% The attacks discussed cannot run in d-mode as the software
% domain would be $\bot$.
% The PDF vulnerability attack is prevented since Acrobat Reader
% runs in d-mode and thus cannot execute the executable file created.
% The DLL attacks are also prevented for the same reasons.
%
% \subsubsection{Using BinInt in Windows}
% 
% When the system is booted, the initial process(es) run in d-mode.
% We enhance  BinInt with an additional
% {\em execution mode policy} (defined by a protected configuration file).
% The execution mode policy is to simplify system usage by predefining
% special cases where the operation of modetrans can be performed
% automatically.
% It specifies the path of an
% executable to have a specific mode and software domain.
% We require that the executable should be b-valid so it is not possible
% to use an arbitrary executable.
% If no such a mapping is found in the policy,
% it inherits the mode and domain of its parent process, e.g. d-mode and $\bot$.
% 
% We have implemented a command-line {\tt modetrans} utility which
% first authenticates the user, i.e. using a password.
% It then executes a user-specified program in a user-specified mode and
% in the case of i-mode, also the specified software domain.
% One could implement a shell extension integrated with the Windows Explorer
% to be more GUI friendly.
% 
% Both the execution mode policy and {\tt modetrans} are used to run an
% executable in a specified mode and domain.
% The policy is preferred if the executable should always run
% in that execution mode and software domain.
% For example, since the Windows auto-updater {\tt wuauclt.exe} should
% always run in i-mode and with sofware domain ``microsoft'', the policy
% should map \path{C:\windows\system32\wuauclt.exe}
% to i-mode and ``microsoft'' software domain.
