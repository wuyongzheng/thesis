\subsection{Implementation and Evaluation}
\label{sec:binint-imp}

We implemented a prototype of BinInt in Windows XP.
It is implemented with a kernel driver that intercepts binary loading,
file modification, process creation and some other operations.
We evaluate the performance of BinInt against
a baseline system with a vanilla unmodified Windows XP SP3.
The benchmarks are run in a VMWare virtual machine 
on a Intel Core 2 Duo 2.33GHz CPU (with one core allocated to VMWare) 
and 512MB memory.
We use the following benchmarks:
software build (building {\tt putty}, a SSH client, with 86 C files) using
the Visual Studio command line interface;
WinRAR archive extraction (extracting the Linux kernel source code);
and the SunSpider JavaScript Benchmark in Firefox.
The first two benchmarks are chosen to test
process creation and file I/O in a realistic setting.
The last benchmark is chosen to be representative of
a more intensive CPU workload.
Each benchmark is run 5 times to obtain
the runtimes (mean and standard deviation) in the base system (without BinInt)
and BinInt in all three modes.

\begin{table}[tb]
\centering
\begin{tabular}{|c|c|c|c|}
\hline
workload & environment & running time & overhead \\ \hline \hline
\multirow{4}{*}{build} &
  base   & $59.0\pm2.0$s & \\ \cline{2-4}
& d-mode & $58.7\pm1.8$s & $-0.5\%$ \\ \cline{2-4}
& t-mode & $59.8\pm1.3$s & $1.4\%$ \\ \cline{2-4}
& i-mode & $60.3\pm1.9$s & $2.2\%$ \\ \hline
\multirow{4}{*}{archive} &
  base   & $41.4\pm1.4$s & \\ \cline{2-4}
& d-mode & $40.0\pm2.2$s & $3.4\%$ \\ \cline{2-4}
& t-mode & $42.2\pm0.8$s & $2.0\%$ \\ \cline{2-4}
& i-mode & $46.1\pm1.1$s & $11.4\%$ \\ \hline
\multirow{4}{*}{javascript} &
  base   & $1257.7\pm21.1$ms & \\ \cline{2-4}
& d-mode & $1257.3\pm25.2$ms & $-0.0\%$ \\ \cline{2-4}
& t-mode & $1249.5\pm26.1$ms & $-0.7\%$ \\ \cline{2-4}
& i-mode & $1247.9\pm15.8$ms & $-0.8\%$ \\ \hline
\end{tabular}
\caption{BinInt overhead}
\label{table:benchmark}
\end{table}

Table~\ref{table:benchmark} shows the runtimes.
Except for the archive extraction test in i-mode,
the average overheads are small.
In most cases, the overhead is smaller than the standard deviation
(caused by environmental factors such as caches, etc., and in some
others it is even negative (slightly faster)).
The results show that with the exception of the archive test, 
the overhead of BinInt is negligible and within timing variation.

The $11.4\%$ overhead in the i-mode archive benchmark
shows the overheads of i-mode when the workload is mostly file creation
and writing of many (small) files.
Although the extracted files are not binaries, BinInt
has to check whether or not they are binaries.
As software installation should not be a frequent task,
we are less concerned with efficiency of
i-mode as long as it is still reasonable.
