\subsection{Security Analysis}
\label{sec:binint-secanal}

We analyse and test the Windows BinInt prototype.
We assume\footnote{
The external monitoring shown in Chapter~\ref{sec:sensor}
can be used to detect some attacks when the assumption is false.}
the operating system kernel together with our BinAuth
kernel driver is trusted.
% The users who have BinAuth administrate privilege are trusted.
% If the system uses BSA, the BSA is trusted.

\subsubsection{Design Analysis}

We first analyse the security provided by the abstraction
provided by the BinInt model.
\medskip

\noindent
Security of {\bf d-mode}: \\
Most processes run in d-mode, thus its security is very
important.
The primary properties in d-mode are: 
b-invalid binaries cannot be loaded; b-valid binary
cannot be modified; and binaries created are b-invalid.
The consequence is that a d-mode process cannot introduce new
binaries to d-mode processes including itself.
This prevents download-and-run or download-and-load attacks.
It is also not possible to delete d-valid binaries.
\medskip

\noindent
Security of {\bf i-mode}: \\
A privilege escalation is required to transition from d-mode to i-mode
using {\tt modetrans}. Thus, malware cannot directly do this, i.e.
our implementation requires user authentication.
However, users might make mistakes or social engineering attacks can
be used.

There are two cases to consider.
If the malware is installed in a new software domain,
the malware cannot modify existing b-valid binary,
thus their integrity is assured.
However, the malware can install new binaries which may be loaded
into existing software as in the DLL hijacking attacks described
in Section~\ref{sec:binint-usageattack}.
The partial order extension can prevent this if the trust level
of the malware is lower than the existing software.
Similarly, the signing extension can also prevent this since the 
malware may not be able to sign the files or it might be denied by
revocation or whitelist.
We remark that as the binary signature database and logs describe
binary usage and loading relationships, malicious behavior can
be detected and also be removed more easily.
% for privilege escalation, we can
% The possibility depends on the existing programs.
% For example, a program may load all DLLs in a specific
% directory as plugin.
% The user is able to discover this using the loading relationship in the binary database.
% In the example case, the user will notice that DLL of the
% newly installed software is loaded by an existing software.
% This can be benign or malicious behaviors.
% Once the user decided that the software to be a malware,
% he can easily remove the binaries of that domain.

If the malware is installed in an existing domain,
the same arguments apply. However, unlike the UAC privilege escalation
mechanism in Windows, the binary database can be used to explain 
whether an executable is connected to the software in the existing domain.
% 
% the malware can modify binaries of the domain.
% The damage that can be caused by the malware depends on the domain.
% For critical domain such as ``microsoft'',
% the malware can affect all programs in the system,
% because all program uses Windows build-in DLLs, which are
% in the ``microsoft'' domain.
% To prevent this from happening, one approach is to use a separate
% password for critical domains so that only system administrator is able to
% {\tt modetrans} to the critical domains.
% In any case, the log of our system can be used to
% identify which binaries are modified by the malware so that
% this can be corrected.
\medskip

\noindent
Security of {\bf t-mode}: \\
Similar to i-mode, t-mode is requires authentication for the 
privilege escalation.
In terms of side effects, it behaves like d-mode, a t-mode process 
cannot modify b-valid binaries,
thus a malicious t-mode process cannot introduce new binaries to d-mode
processes.
However, t-mode processes can load b-invalid binary.
This enables them to introduce new binaries to themselves.
However, if they want to persist in the system (survive after reboot),
they have to lure the user to authenticate and run them in t-mode or i-mode,
as they would not be able to execute in d-mode.


\subsubsection{Implementation Testing}

\TODO{Should I move this to the BinAuth section?}
We test our prototype on various binary execution mechanisms in Windows
in d-mode which is the primary mode
for preventing execution and loading of binaries.

% \noindent
% {\bf Loading b-invalid Binaries}: \\
% Processes running in d-mode cannot load b-invalid binaries.
% We test the prevention mechanism in the prototype:

\begin{itemize}
\item{\bf b-invalid binaries in the {\tt cmd} and Explorer shells} \\
Running a b-invalid binary in the
{\tt cmd} and {\tt Explorer} shells in d-mode fails with an
error 
``{\tt Application} is not a b-valid Win32 application''.
This would also cover cases such as the social engineering file extension
attack.

\item{\bf b-invalid driver loading} \\
It is not possible to load b-invalid drivers.
Using the {\tt net start} command to load a b-invalid driver
fails with the error ``The specified driver is b-invalid''.

\item{\bf b-invalid service loading} \\
B-invalid services cannot be started.
Using the {\tt Services} tab in the Microsoft management console
to start a b-invalid service
fails with the error ``The dependency service or group failed to start''.

\item {\bf b-invalid shell extensions} \\
% In Windows, shell extensions are DLLs that enhance the features of 
% the explorer shell. 
%Common examples are file previews and right click context menus. 
We tested the {\tt Tortoise\-SVN}, {\tt AudioShell} and {\tt FLV} extensions.
{\tt TortoiseSVN} enhances {\tt Explorer} to
interface to the Subversion revision control system with 
a context menu usable by right clicking any 
directory or file in Explorer. 
{\tt AudioShell} is a MP3 tag previewer that is activated by mouse-over
on audio files in Explorer.
{\tt FLV} is a codec which displays Flash video.

If the DLLs for the shell extensions are b-invalid, they do not load.
Thus, the {\tt TortoiseSVN} context menu was missing and
the {\tt AudioShell} mouse-over is no longer shown. 
We tested the use of {\tt Explorer} to open a folder containing
{\tt FLV} (Flash video) files. If the FLV codec, {\tt flvsplitter.ax},
is b-valid, {\tt Explorer} scans the directory, displaying a thumbnail
for each FLV file. When the codec is b-invalid,
{\tt Explorer} does not display any thumbnails and displays the same
icon as files with unrecognized extensions.
We remark that in many cases, users are not
aware that shell extensions are running and may think that the features
are part of {\tt Explorer}.
% E.g. opening a folder containing audio and video files will cause
% {\tt Explorer} to thumbnail all the media files, i.e.
% {\tt Explorer} loads the codec's DLL to extract the thumbnail for video files. 
% An attack which succeeds in installing a malware codec can be
% automatically run by {\tt Explorer}.

\item {\bf b-invalid Browser Helper Objects (BHO)} \\
%BHO's are plugins written to enhance the browser's features and also to provide the
%ability to view certain pages. Common examples of plugins are Flash player, Java runtime,
%Windows Update etc. BHO's are written as DLLs or ActiveX files({\tt ocx}). 
As web browsers are one of the main targets of malware,
it is important to control what binaries are loaded in the browser.
We tested the Flash player control, {\tt Flash10a.ocx}, for Internet Explorer. 
If the control is b-invalid, webpages with flash content do not work as
IE is blocked from loading the Flash control.
However, the rest of the page content is displayed.

\item {\bf PATH manipulation} \\
An attacker may change the {\tt PATH} environment variable or DLL
search order to substitute system executable or DLL with a malicious one.
This is prevented no matter what the search order is, as long as
the malicious executable or DLL is b-invalid.

\end{itemize}


% {\bf Social Engineering} \\
% Some social engineering attacks lure the user into executing malicious
% programs by making use of the flaw of user interfaces.
% For example, Windows Explorer by default hides known file extension.
% User may think the file named {\tt postcard.jpg.exe}
% (which comes from email attachment) to
% be a picture and double click it to view.
% The malicious program may in turn display an image so that the
% user thinks the file is indeed an image.
% This is prevented no matter what file name is, as long as
% the malicious executable is b-invalid.
