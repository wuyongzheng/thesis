\subsection{Security Analysis}
\label{sec:binint-secanal}

We analyse the security provided by BinInt model.
We assume\footnote{
The external monitoring shown in Chapter~\ref{sec:sensor}
can be used to detect some attacks when the assumption is false.}
the operating system kernel together with our BinAuth
kernel driver is trusted.

\noindent
Security of {\bf d-mode}: \\
Most processes run in d-mode, thus its security is very
important.
The primary properties in d-mode are: 
b-invalid binaries cannot be loaded; b-valid binary
cannot be modified; and binaries created are b-invalid.
The consequence is that a d-mode process cannot introduce new
binaries to d-mode processes including itself.
This prevents download-and-run or download-and-load attacks.
It is also not possible to delete d-valid binaries.
\medskip

\noindent
Security of {\bf i-mode}: \\
A privilege escalation is required to transition from d-mode to i-mode
using {\tt modetrans}. Thus, malware cannot directly do this, i.e.
our implementation requires user authentication.
However, users might make mistakes or social engineering attacks can
be used.

There are two cases to consider.
If the malware is installed in a new software domain,
the malware cannot modify existing b-valid binary,
thus their integrity is assured.
However, the malware can install new binaries which may be loaded
into existing software as in the DLL hijacking attacks described
in Section~\ref{sec:binint-usageattack}.
The partial order extension can prevent this if the trust level
of the malware is lower than the existing software.
Similarly, the signing extension can also prevent this since the 
malware may not be able to sign the files or it might be denied by
revocation or whitelist.
We remark that as the binary signature database and logs describe
binary usage and loading relationships, malicious behavior can
be detected and also be removed more easily.
% for privilege escalation, we can
% The possibility depends on the existing programs.
% For example, a program may load all DLLs in a specific
% directory as plugin.
% The user is able to discover this using the loading relationship in the binary database.
% In the example case, the user will notice that DLL of the
% newly installed software is loaded by an existing software.
% This can be benign or malicious behaviors.
% Once the user decided that the software to be a malware,
% he can easily remove the binaries of that domain.

If the malware is installed in an existing domain,
the same arguments apply. However, unlike the UAC privilege escalation
mechanism in Windows, the binary database can be used to explain 
whether an executable is connected to the software in the existing domain.
% 
% the malware can modify binaries of the domain.
% The damage that can be caused by the malware depends on the domain.
% For critical domain such as ``microsoft'',
% the malware can affect all programs in the system,
% because all program uses Windows build-in DLLs, which are
% in the ``microsoft'' domain.
% To prevent this from happening, one approach is to use a separate
% password for critical domains so that only system administrator is able to
% {\tt modetrans} to the critical domains.
% In any case, the log of our system can be used to
% identify which binaries are modified by the malware so that
% this can be corrected.
\medskip

\noindent
Security of {\bf t-mode}: \\
Similar to i-mode, t-mode is requires authentication for the 
privilege escalation.
In terms of side effects, it behaves like d-mode, a t-mode process 
cannot modify b-valid binaries,
thus a malicious t-mode process cannot introduce new binaries to d-mode
processes.
However, t-mode processes can load b-invalid binary.
This enables them to introduce new binaries to themselves.
However, if they want to persist in the system (survive after reboot),
they have to lure the user to authenticate and run them in t-mode or i-mode,
as they would not be able to execute in d-mode.
