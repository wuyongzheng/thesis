% \subsection{Introduction}

Securing Windows is a challenge because of its large
attack surface which can lead to many ways where
binaries can be loaded and subsequently executed.
Software in Windows also has many interactions with each other.
These interactions are often not well understood or specified.
Software is typically closed source, in binary form only.

In this section, we use software to mean binary files which are either
executables or Dynamic Link Libraries (DLLs).
We have two goals. Firstly, the security objective 
is to ensure the use of trusted binaries and maintain their integrity.
This protects against attacks which attempt to modify or remove binaries 
commonly used in malware or denial of service attacks.
We prevent malware attacks which rely on
getting a malicious executable to be run or
a malicious DLL to load.
In addition, we prevent the malware from being persistent.

Secondly, the security mechanism needs to be usable with typical
software in binary form.
It needs to be compatible with the software life cycle of software in the
system, i.e.
software may be installed, then used, then updated, and 
finally uninstalled. 
Portions of installed software, e.g. DLLs,
may be shared with other software. 
Ideally, the security mechanism should be (mostly) transparent to 
this software life cycle.
While it may not be possible to be completely transparent/compatible 
for arbitrary binaries, it should support the mechanisms used 
by typical software.
We remark that as we are concerned with
closed source and binaries,
approaches which change the process of installation/update,
usually requiring (some source code), are not relevant.
Clearly, there is a tradeoff between security and usability --
we seek a practical middle ground with improved security while
being compatible with typical software.

In this section, we present a new security model
for binaries called {\em BinInt}.
We implement a prototype of BinInt for Windows 
which meets our security objectives yet allows
for software to be changed as part of the software lifecyle.
It provides a number of modes for operation: default, install 
and temporary trusted modes.
BinInt provides protection (in default mode) against a broad range of 
binary-based attacks.
Our system is efficient and has negligible overhead in default and temporary
trusted modes. 
In install mode, our benchmarks
show $\sim$12\% overhead which is reasonable since
the use of install mode is infrequent.

The section is organized as follows.
Mechanisms for loading binaries in
Windows and binary-based attacks are reviewed in Sec.~\ref{sec:problems}.
Sec.~\ref{sec:prevworks} discusses related work.
Our binary security model is described in Sec.~\ref{sec:binint}.
Implementation and evaluation of BinInt is in
Sec.~\ref{sec:imp} and Sec.~\ref{sec:conc} concludes.

