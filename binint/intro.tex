% \subsection{Introduction}

The previous section showed BinAuth, a reliable and efficient system
to authenticate binaries in Windows.
It is designed for enviroments where the binaries are relatively static,
i.e. the binaries rarely change or the changes are well managed by
the system administrator.
However, in systems such as personal computers, it is common for binaries
to change.
In particular, users can install or uninstall software;
The software can perform automatic updates periodically;
and some software creates temporary binaries.

In this section, we first look at the dynamics of binaries and compare
it with attacks in Section~\ref{sec:binint-usageattack}.
We then discuss the related work in Section~~\ref{sec:binint-prevworks}.
After that, we present a new security model for binaries called
{\em BinInt} in Section~\ref{sec:binint-model}.
It provides a number of modes for operation: default, install
and temporary trusted modes.
BinInt provides protection (in default mode) against a broad range of
binary-based attacks.
In Section~\ref{sec:binint-imp}, we show the implementation and evaluation.
Our system is efficient and has negligible overhead in default and temporary
trusted modes. 
In install mode, our benchmarks
show $\sim$12\% overhead which is reasonable since
the use of install mode is infrequent.
