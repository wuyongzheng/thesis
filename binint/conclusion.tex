\subsection{Discussion and Conclusion}
\label{sec:binint-conc}

BinInt focuses on binaries.
However, securing only binaries may not be sufficient.
For example, the attacker can modify Java class files to
change the behaviour of java programs.
The same applies to shell scripts and configuration files.
We can generalize BinInt to protect the integrity of any kind of file.
The difficulty is that data files are much more dynamic than binaries.
Unlike binaries which are only changed during install, update and removal
of the software, data files are usually changed much more frequently.
This makes direct use of d-mode for data files unusable.
More fine grained policies can be applied for data files.
This is currently work in progress.

We have presented a binary security model which caters to the
dynamic use of binaries within the software life cycle
while protecting against attacks in default mode
and giving isolation between software domains in install mode.
Our prototype is efficient and usable while
protecting a broad range of binary loading/execution mechanisms in Windows.
We found our system to be mostly transparent in usage on typical
Windows software throughout its software lifecycle.
Thus, BinInt is a practical solution
which gives a good tradeoff between usability and security
to protect binaries on Windows.
Our model can also be combined with other security mechanisms.

